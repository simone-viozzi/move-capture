\documentclass[10pt,a4paper]{article}

\usepackage[italian]{babel}
\usepackage{amsmath}
\usepackage{amsfonts}
\usepackage{amssymb}

\usepackage[left=1cm,right=1cm,top=1cm,bottom=2cm]{geometry}

\usepackage{txfonts}
\usepackage[T1]{fontenc}
\usepackage[utf8]{inputenc}

\usepackage{titlesec}
\setcounter{secnumdepth}{4}
\titleformat{\subsection}{\normalfont\fontsize{13}{13}\bfseries}{\thesubsection}{1em}{}
\titleformat{\subsubsection}{\normalfont\fontsize{11}{11}\bfseries}{\thesubsubsection}{1em}{}
\titleformat{\paragraph}{\normalfont\normalsize\bfseries}{\theparagraph}{1em}{}
\titlespacing*{\paragraph}{0pt}{3.25ex plus 1ex minus .2ex}{1.5ex plus .2ex}

%per le immagini
\usepackage{graphicx}
\usepackage{subcaption}
\usepackage{wrapfig}

%per i link
\usepackage{hyperref} 

%%%%%%%% per il codice c++
\usepackage{textcomp}
\usepackage{listings}          % for creating language style
\usepackage{listingsutf8}
 %%%%%%%%%%%%%%%%%%%%%%%%%%%%%%%%%%%%%%%%%%%%%%%%%%%%%%%%%%%%%%%%%%%%%%%%%%%%%%%% 
%%% ~ Arduino Language - Arduino IDE Colors ~                                  %%%
%%%                                                                            %%%
%%% Kyle Rocha-Brownell | 10/2/2017 | No Licence                               %%%
%%% -------------------------------------------------------------------------- %%%
%%%                                                                            %%%
%%% Place this file in your working directory (next to the latex file you're   %%%
%%% working on).  To add it to your project, place:                            %%%
%%%     %%%%%%%%%%%%%%%%%%%%%%%%%%%%%%%%%%%%%%%%%%%%%%%%%%%%%%%%%%%%%%%%%%%%%%%%%%%%%%%% 
%%% ~ Arduino Language - Arduino IDE Colors ~                                  %%%
%%%                                                                            %%%
%%% Kyle Rocha-Brownell | 10/2/2017 | No Licence                               %%%
%%% -------------------------------------------------------------------------- %%%
%%%                                                                            %%%
%%% Place this file in your working directory (next to the latex file you're   %%%
%%% working on).  To add it to your project, place:                            %%%
%%%     %%%%%%%%%%%%%%%%%%%%%%%%%%%%%%%%%%%%%%%%%%%%%%%%%%%%%%%%%%%%%%%%%%%%%%%%%%%%%%%% 
%%% ~ Arduino Language - Arduino IDE Colors ~                                  %%%
%%%                                                                            %%%
%%% Kyle Rocha-Brownell | 10/2/2017 | No Licence                               %%%
%%% -------------------------------------------------------------------------- %%%
%%%                                                                            %%%
%%% Place this file in your working directory (next to the latex file you're   %%%
%%% working on).  To add it to your project, place:                            %%%
%%%    \input{arduinoLanguage.tex}                                             %%%
%%% somewhere before \begin{document} in your latex file.                      %%%
%%%                                                                            %%%
%%% In your document, place your arduino code between:                         %%%
%%%   \begin{lstlisting}[language=Arduino]                                     %%%
%%% and:                                                                       %%%
%%%   \end{lstlisting}                                                         %%%
%%%                                                                            %%%
%%% Or create your own style to add non-built-in functions and variables.      %%%
%%%                                                                            %%%
 %%%%%%%%%%%%%%%%%%%%%%%%%%%%%%%%%%%%%%%%%%%%%%%%%%%%%%%%%%%%%%%%%%%%%%%%%%%%%%%% 

\usepackage{color}
\usepackage{listings}    
\usepackage{courier}
\usepackage{listingsutf8}

%%% Define Custom IDE Colors %%%
\definecolor{arduinoGreen}    {rgb} {0.17, 0.43, 0.01}
\definecolor{arduinoGrey}     {rgb} {0.47, 0.47, 0.33}
\definecolor{arduinoOrange}   {rgb} {0.8 , 0.4 , 0   }
\definecolor{arduinoBlue}     {rgb} {0.01, 0.61, 0.98}
\definecolor{arduinoDarkBlue} {rgb} {0.0 , 0.2 , 0.5 }
\definecolor{light-gray}{gray}{0.95}

%%% Define Arduino Language %%%
\lstdefinelanguage{Arduino}{
  language=C++, % begin with default C++ settings 
%
%
  %%% Keyword Color Group 1 %%%  (called KEYWORD3 by arduino)
  keywordstyle=\color{arduinoGreen},   
  deletekeywords={  % remove all arduino keywords that might be in c++
                break, case, override, final, continue, default, do, else, for, 
                if, return, goto, switch, throw, try, while, setup, loop, export, 
                not, or, and, xor, include, define, elif, else, error, if, ifdef, 
                ifndef, pragma, warning,
                HIGH, LOW, INPUT, INPUT_PULLUP, OUTPUT, DEC, BIN, HEX, OCT, PI, 
                HALF_PI, TWO_PI, LSBFIRST, MSBFIRST, CHANGE, FALLING, RISING, 
                DEFAULT, EXTERNAL, INTERNAL, INTERNAL1V1, INTERNAL2V56, LED_BUILTIN, 
                LED_BUILTIN_RX, LED_BUILTIN_TX, DIGITAL_MESSAGE, FIRMATA_STRING, 
                ANALOG_MESSAGE, REPORT_DIGITAL, REPORT_ANALOG, SET_PIN_MODE, 
                SYSTEM_RESET, SYSEX_START, auto, int8_t, int16_t, int32_t, int64_t, 
                uint8_t, uint16_t, uint32_t, uint64_t, char16_t, char32_t, operator, 
                enum, delete, bool, boolean, byte, char, const, false, float, double, 
                null, NULL, int, long, new, private, protected, public, short, 
                signed, static, volatile, String, void, true, unsigned, word, array, 
                sizeof, dynamic_cast, typedef, const_cast, struct, static_cast, union, 
                friend, extern, class, reinterpret_cast, register, explicit, inline, 
                _Bool, complex, _Complex, _Imaginary, atomic_bool, atomic_char, 
                atomic_schar, atomic_uchar, atomic_short, atomic_ushort, atomic_int, 
                atomic_uint, atomic_long, atomic_ulong, atomic_llong, atomic_ullong, 
                virtual, PROGMEM,
                Serial, Serial1, Serial2, Serial3, SerialUSB, Keyboard, Mouse,
                abs, acos, asin, atan, atan2, ceil, constrain, cos, degrees, exp, 
                floor, log, map, max, min, radians, random, randomSeed, round, sin, 
                sq, sqrt, tan, pow, bitRead, bitWrite, bitSet, bitClear, bit, 
                highByte, lowByte, analogReference, analogRead, 
                analogReadResolution, analogWrite, analogWriteResolution, 
                attachInterrupt, detachInterrupt, digitalPinToInterrupt, delay, 
                delayMicroseconds, digitalWrite, digitalRead, interrupts, millis, 
                micros, noInterrupts, noTone, pinMode, pulseIn, pulseInLong, shiftIn, 
                shiftOut, tone, yield, Stream, begin, end, peek, read, print, 
                println, available, availableForWrite, flush, setTimeout, find, 
                findUntil, parseInt, parseFloat, readBytes, readBytesUntil, readString, 
                readStringUntil, trim, toUpperCase, toLowerCase, charAt, compareTo, 
                concat, endsWith, startsWith, equals, equalsIgnoreCase, getBytes, 
                indexOf, lastIndexOf, length, replace, setCharAt, substring, 
                toCharArray, toInt, press, release, releaseAll, accept, click, move, 
                isPressed, isAlphaNumeric, isAlpha, isAscii, isWhitespace, isControl, 
                isDigit, isGraph, isLowerCase, isPrintable, isPunct, isSpace, 
                isUpperCase, isHexadecimalDigit, 
                }, 
  morekeywords={   % add arduino structures to group 1
                break, case, override, final, continue, default, do, else, for, 
                if, return, goto, switch, throw, try, while, setup, loop, export, 
                not, or, and, xor, include, define, elif, else, error, if, ifdef, 
                ifndef, pragma, warning,
                }, 
% 
%
  %%% Keyword Color Group 2 %%%  (called LITERAL1 by arduino)
  keywordstyle=[2]\color{arduinoBlue},   
  keywords=[2]{   % add variables and dataTypes as 2nd group  
                HIGH, LOW, INPUT, INPUT_PULLUP, OUTPUT, DEC, BIN, HEX, OCT, PI, 
                HALF_PI, TWO_PI, LSBFIRST, MSBFIRST, CHANGE, FALLING, RISING, 
                DEFAULT, EXTERNAL, INTERNAL, INTERNAL1V1, INTERNAL2V56, LED_BUILTIN, 
                LED_BUILTIN_RX, LED_BUILTIN_TX, DIGITAL_MESSAGE, FIRMATA_STRING, 
                ANALOG_MESSAGE, REPORT_DIGITAL, REPORT_ANALOG, SET_PIN_MODE, 
                SYSTEM_RESET, SYSEX_START, auto, int8_t, int16_t, int32_t, int64_t, 
                uint8_t, uint16_t, uint32_t, uint64_t, char16_t, char32_t, operator, 
                enum, delete, bool, boolean, byte, char, const, false, float, double, 
                null, NULL, int, long, new, private, protected, public, short, 
                signed, static, volatile, String, void, true, unsigned, word, array, 
                sizeof, dynamic_cast, typedef, const_cast, struct, static_cast, union, 
                friend, extern, class, reinterpret_cast, register, explicit, inline, 
                _Bool, complex, _Complex, _Imaginary, atomic_bool, atomic_char, 
                atomic_schar, atomic_uchar, atomic_short, atomic_ushort, atomic_int, 
                atomic_uint, atomic_long, atomic_ulong, atomic_llong, atomic_ullong, 
                virtual, PROGMEM,
                },  
% 
%
  %%% Keyword Color Group 3 %%%  (called KEYWORD1 by arduino)
  keywordstyle=[3]\bfseries\color{arduinoOrange},
  keywords=[3]{  % add built-in functions as a 3rd group
                Serial, Serial1, Serial2, Serial3, SerialUSB, Keyboard, Mouse,
                },      
%
%
  %%% Keyword Color Group 4 %%%  (called KEYWORD2 by arduino)
  keywordstyle=[4]\color{arduinoOrange},
  keywords=[4]{  % add more built-in functions as a 4th group
                abs, acos, asin, atan, atan2, ceil, constrain, cos, degrees, exp, 
                floor, log, map, max, min, radians, random, randomSeed, round, sin, 
                sq, sqrt, tan, pow, bitRead, bitWrite, bitSet, bitClear, bit, 
                highByte, lowByte, analogReference, analogRead, 
                analogReadResolution, analogWrite, analogWriteResolution, 
                attachInterrupt, detachInterrupt, digitalPinToInterrupt, delay, 
                delayMicroseconds, digitalWrite, digitalRead, interrupts, millis, 
                micros, noInterrupts, noTone, pinMode, pulseIn, pulseInLong, shiftIn, 
                shiftOut, tone, yield, Stream, begin, end, peek, read, print, 
                println, available, availableForWrite, flush, setTimeout, find, 
                findUntil, parseInt, parseFloat, readBytes, readBytesUntil, readString, 
                readStringUntil, trim, toUpperCase, toLowerCase, charAt, compareTo, 
                concat, endsWith, startsWith, equals, equalsIgnoreCase, getBytes, 
                indexOf, lastIndexOf, length, replace, setCharAt, substring, 
                toCharArray, toInt, press, release, releaseAll, accept, click, move, 
                isPressed, isAlphaNumeric, isAlpha, isAscii, isWhitespace, isControl, 
                isDigit, isGraph, isLowerCase, isPrintable, isPunct, isSpace, 
                isUpperCase, isHexadecimalDigit, 
                },      
%
  extendedchars=true,
%
  %%% Set Other Colors %%%
  stringstyle=\color{arduinoDarkBlue},
  showstringspaces=false,  
  commentstyle=\color{arduinoGrey},    
%          
%   
  %%%% Line Numbering %%%%
  numbers=left,                    
  numbersep=5pt,                   
  numberstyle=\color{arduinoGrey},    
  %stepnumber=2,                      % show every 2 line numbers
%
%
  %%%% Code Box Style %%%%
  breaklines=true,                    % wordwrapping
  tabsize=2,         
  basicstyle=\fontsize{9}{11}\ttfamily,
  backgroundcolor=\color{light-gray},
  xleftmargin=.35in
}                                             %%%
%%% somewhere before \begin{document} in your latex file.                      %%%
%%%                                                                            %%%
%%% In your document, place your arduino code between:                         %%%
%%%   \begin{lstlisting}[language=Arduino]                                     %%%
%%% and:                                                                       %%%
%%%   \end{lstlisting}                                                         %%%
%%%                                                                            %%%
%%% Or create your own style to add non-built-in functions and variables.      %%%
%%%                                                                            %%%
 %%%%%%%%%%%%%%%%%%%%%%%%%%%%%%%%%%%%%%%%%%%%%%%%%%%%%%%%%%%%%%%%%%%%%%%%%%%%%%%% 

\usepackage{color}
\usepackage{listings}    
\usepackage{courier}
\usepackage{listingsutf8}

%%% Define Custom IDE Colors %%%
\definecolor{arduinoGreen}    {rgb} {0.17, 0.43, 0.01}
\definecolor{arduinoGrey}     {rgb} {0.47, 0.47, 0.33}
\definecolor{arduinoOrange}   {rgb} {0.8 , 0.4 , 0   }
\definecolor{arduinoBlue}     {rgb} {0.01, 0.61, 0.98}
\definecolor{arduinoDarkBlue} {rgb} {0.0 , 0.2 , 0.5 }
\definecolor{light-gray}{gray}{0.95}

%%% Define Arduino Language %%%
\lstdefinelanguage{Arduino}{
  language=C++, % begin with default C++ settings 
%
%
  %%% Keyword Color Group 1 %%%  (called KEYWORD3 by arduino)
  keywordstyle=\color{arduinoGreen},   
  deletekeywords={  % remove all arduino keywords that might be in c++
                break, case, override, final, continue, default, do, else, for, 
                if, return, goto, switch, throw, try, while, setup, loop, export, 
                not, or, and, xor, include, define, elif, else, error, if, ifdef, 
                ifndef, pragma, warning,
                HIGH, LOW, INPUT, INPUT_PULLUP, OUTPUT, DEC, BIN, HEX, OCT, PI, 
                HALF_PI, TWO_PI, LSBFIRST, MSBFIRST, CHANGE, FALLING, RISING, 
                DEFAULT, EXTERNAL, INTERNAL, INTERNAL1V1, INTERNAL2V56, LED_BUILTIN, 
                LED_BUILTIN_RX, LED_BUILTIN_TX, DIGITAL_MESSAGE, FIRMATA_STRING, 
                ANALOG_MESSAGE, REPORT_DIGITAL, REPORT_ANALOG, SET_PIN_MODE, 
                SYSTEM_RESET, SYSEX_START, auto, int8_t, int16_t, int32_t, int64_t, 
                uint8_t, uint16_t, uint32_t, uint64_t, char16_t, char32_t, operator, 
                enum, delete, bool, boolean, byte, char, const, false, float, double, 
                null, NULL, int, long, new, private, protected, public, short, 
                signed, static, volatile, String, void, true, unsigned, word, array, 
                sizeof, dynamic_cast, typedef, const_cast, struct, static_cast, union, 
                friend, extern, class, reinterpret_cast, register, explicit, inline, 
                _Bool, complex, _Complex, _Imaginary, atomic_bool, atomic_char, 
                atomic_schar, atomic_uchar, atomic_short, atomic_ushort, atomic_int, 
                atomic_uint, atomic_long, atomic_ulong, atomic_llong, atomic_ullong, 
                virtual, PROGMEM,
                Serial, Serial1, Serial2, Serial3, SerialUSB, Keyboard, Mouse,
                abs, acos, asin, atan, atan2, ceil, constrain, cos, degrees, exp, 
                floor, log, map, max, min, radians, random, randomSeed, round, sin, 
                sq, sqrt, tan, pow, bitRead, bitWrite, bitSet, bitClear, bit, 
                highByte, lowByte, analogReference, analogRead, 
                analogReadResolution, analogWrite, analogWriteResolution, 
                attachInterrupt, detachInterrupt, digitalPinToInterrupt, delay, 
                delayMicroseconds, digitalWrite, digitalRead, interrupts, millis, 
                micros, noInterrupts, noTone, pinMode, pulseIn, pulseInLong, shiftIn, 
                shiftOut, tone, yield, Stream, begin, end, peek, read, print, 
                println, available, availableForWrite, flush, setTimeout, find, 
                findUntil, parseInt, parseFloat, readBytes, readBytesUntil, readString, 
                readStringUntil, trim, toUpperCase, toLowerCase, charAt, compareTo, 
                concat, endsWith, startsWith, equals, equalsIgnoreCase, getBytes, 
                indexOf, lastIndexOf, length, replace, setCharAt, substring, 
                toCharArray, toInt, press, release, releaseAll, accept, click, move, 
                isPressed, isAlphaNumeric, isAlpha, isAscii, isWhitespace, isControl, 
                isDigit, isGraph, isLowerCase, isPrintable, isPunct, isSpace, 
                isUpperCase, isHexadecimalDigit, 
                }, 
  morekeywords={   % add arduino structures to group 1
                break, case, override, final, continue, default, do, else, for, 
                if, return, goto, switch, throw, try, while, setup, loop, export, 
                not, or, and, xor, include, define, elif, else, error, if, ifdef, 
                ifndef, pragma, warning,
                }, 
% 
%
  %%% Keyword Color Group 2 %%%  (called LITERAL1 by arduino)
  keywordstyle=[2]\color{arduinoBlue},   
  keywords=[2]{   % add variables and dataTypes as 2nd group  
                HIGH, LOW, INPUT, INPUT_PULLUP, OUTPUT, DEC, BIN, HEX, OCT, PI, 
                HALF_PI, TWO_PI, LSBFIRST, MSBFIRST, CHANGE, FALLING, RISING, 
                DEFAULT, EXTERNAL, INTERNAL, INTERNAL1V1, INTERNAL2V56, LED_BUILTIN, 
                LED_BUILTIN_RX, LED_BUILTIN_TX, DIGITAL_MESSAGE, FIRMATA_STRING, 
                ANALOG_MESSAGE, REPORT_DIGITAL, REPORT_ANALOG, SET_PIN_MODE, 
                SYSTEM_RESET, SYSEX_START, auto, int8_t, int16_t, int32_t, int64_t, 
                uint8_t, uint16_t, uint32_t, uint64_t, char16_t, char32_t, operator, 
                enum, delete, bool, boolean, byte, char, const, false, float, double, 
                null, NULL, int, long, new, private, protected, public, short, 
                signed, static, volatile, String, void, true, unsigned, word, array, 
                sizeof, dynamic_cast, typedef, const_cast, struct, static_cast, union, 
                friend, extern, class, reinterpret_cast, register, explicit, inline, 
                _Bool, complex, _Complex, _Imaginary, atomic_bool, atomic_char, 
                atomic_schar, atomic_uchar, atomic_short, atomic_ushort, atomic_int, 
                atomic_uint, atomic_long, atomic_ulong, atomic_llong, atomic_ullong, 
                virtual, PROGMEM,
                },  
% 
%
  %%% Keyword Color Group 3 %%%  (called KEYWORD1 by arduino)
  keywordstyle=[3]\bfseries\color{arduinoOrange},
  keywords=[3]{  % add built-in functions as a 3rd group
                Serial, Serial1, Serial2, Serial3, SerialUSB, Keyboard, Mouse,
                },      
%
%
  %%% Keyword Color Group 4 %%%  (called KEYWORD2 by arduino)
  keywordstyle=[4]\color{arduinoOrange},
  keywords=[4]{  % add more built-in functions as a 4th group
                abs, acos, asin, atan, atan2, ceil, constrain, cos, degrees, exp, 
                floor, log, map, max, min, radians, random, randomSeed, round, sin, 
                sq, sqrt, tan, pow, bitRead, bitWrite, bitSet, bitClear, bit, 
                highByte, lowByte, analogReference, analogRead, 
                analogReadResolution, analogWrite, analogWriteResolution, 
                attachInterrupt, detachInterrupt, digitalPinToInterrupt, delay, 
                delayMicroseconds, digitalWrite, digitalRead, interrupts, millis, 
                micros, noInterrupts, noTone, pinMode, pulseIn, pulseInLong, shiftIn, 
                shiftOut, tone, yield, Stream, begin, end, peek, read, print, 
                println, available, availableForWrite, flush, setTimeout, find, 
                findUntil, parseInt, parseFloat, readBytes, readBytesUntil, readString, 
                readStringUntil, trim, toUpperCase, toLowerCase, charAt, compareTo, 
                concat, endsWith, startsWith, equals, equalsIgnoreCase, getBytes, 
                indexOf, lastIndexOf, length, replace, setCharAt, substring, 
                toCharArray, toInt, press, release, releaseAll, accept, click, move, 
                isPressed, isAlphaNumeric, isAlpha, isAscii, isWhitespace, isControl, 
                isDigit, isGraph, isLowerCase, isPrintable, isPunct, isSpace, 
                isUpperCase, isHexadecimalDigit, 
                },      
%
  extendedchars=true,
%
  %%% Set Other Colors %%%
  stringstyle=\color{arduinoDarkBlue},
  showstringspaces=false,  
  commentstyle=\color{arduinoGrey},    
%          
%   
  %%%% Line Numbering %%%%
  numbers=left,                    
  numbersep=5pt,                   
  numberstyle=\color{arduinoGrey},    
  %stepnumber=2,                      % show every 2 line numbers
%
%
  %%%% Code Box Style %%%%
  breaklines=true,                    % wordwrapping
  tabsize=2,         
  basicstyle=\fontsize{9}{11}\ttfamily,
  backgroundcolor=\color{light-gray},
  xleftmargin=.35in
}                                             %%%
%%% somewhere before \begin{document} in your latex file.                      %%%
%%%                                                                            %%%
%%% In your document, place your arduino code between:                         %%%
%%%   \begin{lstlisting}[language=Arduino]                                     %%%
%%% and:                                                                       %%%
%%%   \end{lstlisting}                                                         %%%
%%%                                                                            %%%
%%% Or create your own style to add non-built-in functions and variables.      %%%
%%%                                                                            %%%
 %%%%%%%%%%%%%%%%%%%%%%%%%%%%%%%%%%%%%%%%%%%%%%%%%%%%%%%%%%%%%%%%%%%%%%%%%%%%%%%% 

\usepackage{color}
\usepackage{listings}    
\usepackage{courier}
\usepackage{listingsutf8}

%%% Define Custom IDE Colors %%%
\definecolor{arduinoGreen}    {rgb} {0.17, 0.43, 0.01}
\definecolor{arduinoGrey}     {rgb} {0.47, 0.47, 0.33}
\definecolor{arduinoOrange}   {rgb} {0.8 , 0.4 , 0   }
\definecolor{arduinoBlue}     {rgb} {0.01, 0.61, 0.98}
\definecolor{arduinoDarkBlue} {rgb} {0.0 , 0.2 , 0.5 }
\definecolor{light-gray}{gray}{0.95}

%%% Define Arduino Language %%%
\lstdefinelanguage{Arduino}{
  language=C++, % begin with default C++ settings 
%
%
  %%% Keyword Color Group 1 %%%  (called KEYWORD3 by arduino)
  keywordstyle=\color{arduinoGreen},   
  deletekeywords={  % remove all arduino keywords that might be in c++
                break, case, override, final, continue, default, do, else, for, 
                if, return, goto, switch, throw, try, while, setup, loop, export, 
                not, or, and, xor, include, define, elif, else, error, if, ifdef, 
                ifndef, pragma, warning,
                HIGH, LOW, INPUT, INPUT_PULLUP, OUTPUT, DEC, BIN, HEX, OCT, PI, 
                HALF_PI, TWO_PI, LSBFIRST, MSBFIRST, CHANGE, FALLING, RISING, 
                DEFAULT, EXTERNAL, INTERNAL, INTERNAL1V1, INTERNAL2V56, LED_BUILTIN, 
                LED_BUILTIN_RX, LED_BUILTIN_TX, DIGITAL_MESSAGE, FIRMATA_STRING, 
                ANALOG_MESSAGE, REPORT_DIGITAL, REPORT_ANALOG, SET_PIN_MODE, 
                SYSTEM_RESET, SYSEX_START, auto, int8_t, int16_t, int32_t, int64_t, 
                uint8_t, uint16_t, uint32_t, uint64_t, char16_t, char32_t, operator, 
                enum, delete, bool, boolean, byte, char, const, false, float, double, 
                null, NULL, int, long, new, private, protected, public, short, 
                signed, static, volatile, String, void, true, unsigned, word, array, 
                sizeof, dynamic_cast, typedef, const_cast, struct, static_cast, union, 
                friend, extern, class, reinterpret_cast, register, explicit, inline, 
                _Bool, complex, _Complex, _Imaginary, atomic_bool, atomic_char, 
                atomic_schar, atomic_uchar, atomic_short, atomic_ushort, atomic_int, 
                atomic_uint, atomic_long, atomic_ulong, atomic_llong, atomic_ullong, 
                virtual, PROGMEM,
                Serial, Serial1, Serial2, Serial3, SerialUSB, Keyboard, Mouse,
                abs, acos, asin, atan, atan2, ceil, constrain, cos, degrees, exp, 
                floor, log, map, max, min, radians, random, randomSeed, round, sin, 
                sq, sqrt, tan, pow, bitRead, bitWrite, bitSet, bitClear, bit, 
                highByte, lowByte, analogReference, analogRead, 
                analogReadResolution, analogWrite, analogWriteResolution, 
                attachInterrupt, detachInterrupt, digitalPinToInterrupt, delay, 
                delayMicroseconds, digitalWrite, digitalRead, interrupts, millis, 
                micros, noInterrupts, noTone, pinMode, pulseIn, pulseInLong, shiftIn, 
                shiftOut, tone, yield, Stream, begin, end, peek, read, print, 
                println, available, availableForWrite, flush, setTimeout, find, 
                findUntil, parseInt, parseFloat, readBytes, readBytesUntil, readString, 
                readStringUntil, trim, toUpperCase, toLowerCase, charAt, compareTo, 
                concat, endsWith, startsWith, equals, equalsIgnoreCase, getBytes, 
                indexOf, lastIndexOf, length, replace, setCharAt, substring, 
                toCharArray, toInt, press, release, releaseAll, accept, click, move, 
                isPressed, isAlphaNumeric, isAlpha, isAscii, isWhitespace, isControl, 
                isDigit, isGraph, isLowerCase, isPrintable, isPunct, isSpace, 
                isUpperCase, isHexadecimalDigit, 
                }, 
  morekeywords={   % add arduino structures to group 1
                break, case, override, final, continue, default, do, else, for, 
                if, return, goto, switch, throw, try, while, setup, loop, export, 
                not, or, and, xor, include, define, elif, else, error, if, ifdef, 
                ifndef, pragma, warning,
                }, 
% 
%
  %%% Keyword Color Group 2 %%%  (called LITERAL1 by arduino)
  keywordstyle=[2]\color{arduinoBlue},   
  keywords=[2]{   % add variables and dataTypes as 2nd group  
                HIGH, LOW, INPUT, INPUT_PULLUP, OUTPUT, DEC, BIN, HEX, OCT, PI, 
                HALF_PI, TWO_PI, LSBFIRST, MSBFIRST, CHANGE, FALLING, RISING, 
                DEFAULT, EXTERNAL, INTERNAL, INTERNAL1V1, INTERNAL2V56, LED_BUILTIN, 
                LED_BUILTIN_RX, LED_BUILTIN_TX, DIGITAL_MESSAGE, FIRMATA_STRING, 
                ANALOG_MESSAGE, REPORT_DIGITAL, REPORT_ANALOG, SET_PIN_MODE, 
                SYSTEM_RESET, SYSEX_START, auto, int8_t, int16_t, int32_t, int64_t, 
                uint8_t, uint16_t, uint32_t, uint64_t, char16_t, char32_t, operator, 
                enum, delete, bool, boolean, byte, char, const, false, float, double, 
                null, NULL, int, long, new, private, protected, public, short, 
                signed, static, volatile, String, void, true, unsigned, word, array, 
                sizeof, dynamic_cast, typedef, const_cast, struct, static_cast, union, 
                friend, extern, class, reinterpret_cast, register, explicit, inline, 
                _Bool, complex, _Complex, _Imaginary, atomic_bool, atomic_char, 
                atomic_schar, atomic_uchar, atomic_short, atomic_ushort, atomic_int, 
                atomic_uint, atomic_long, atomic_ulong, atomic_llong, atomic_ullong, 
                virtual, PROGMEM,
                },  
% 
%
  %%% Keyword Color Group 3 %%%  (called KEYWORD1 by arduino)
  keywordstyle=[3]\bfseries\color{arduinoOrange},
  keywords=[3]{  % add built-in functions as a 3rd group
                Serial, Serial1, Serial2, Serial3, SerialUSB, Keyboard, Mouse,
                },      
%
%
  %%% Keyword Color Group 4 %%%  (called KEYWORD2 by arduino)
  keywordstyle=[4]\color{arduinoOrange},
  keywords=[4]{  % add more built-in functions as a 4th group
                abs, acos, asin, atan, atan2, ceil, constrain, cos, degrees, exp, 
                floor, log, map, max, min, radians, random, randomSeed, round, sin, 
                sq, sqrt, tan, pow, bitRead, bitWrite, bitSet, bitClear, bit, 
                highByte, lowByte, analogReference, analogRead, 
                analogReadResolution, analogWrite, analogWriteResolution, 
                attachInterrupt, detachInterrupt, digitalPinToInterrupt, delay, 
                delayMicroseconds, digitalWrite, digitalRead, interrupts, millis, 
                micros, noInterrupts, noTone, pinMode, pulseIn, pulseInLong, shiftIn, 
                shiftOut, tone, yield, Stream, begin, end, peek, read, print, 
                println, available, availableForWrite, flush, setTimeout, find, 
                findUntil, parseInt, parseFloat, readBytes, readBytesUntil, readString, 
                readStringUntil, trim, toUpperCase, toLowerCase, charAt, compareTo, 
                concat, endsWith, startsWith, equals, equalsIgnoreCase, getBytes, 
                indexOf, lastIndexOf, length, replace, setCharAt, substring, 
                toCharArray, toInt, press, release, releaseAll, accept, click, move, 
                isPressed, isAlphaNumeric, isAlpha, isAscii, isWhitespace, isControl, 
                isDigit, isGraph, isLowerCase, isPrintable, isPunct, isSpace, 
                isUpperCase, isHexadecimalDigit, 
                },      
%
  extendedchars=true,
%
  %%% Set Other Colors %%%
  stringstyle=\color{arduinoDarkBlue},
  showstringspaces=false,  
  commentstyle=\color{arduinoGrey},    
%          
%   
  %%%% Line Numbering %%%%
  numbers=left,                    
  numbersep=5pt,                   
  numberstyle=\color{arduinoGrey},    
  %stepnumber=2,                      % show every 2 line numbers
%
%
  %%%% Code Box Style %%%%
  breaklines=true,                    % wordwrapping
  tabsize=2,         
  basicstyle=\fontsize{9}{11}\ttfamily,
  backgroundcolor=\color{light-gray},
  xleftmargin=.35in
}    % adds the arduino language listing
\definecolor{commentgreen}    {RGB}{2,112,10}
\definecolor{eminence}        {RGB}{108,48,130}
\definecolor{weborange}       {RGB}{255,165,0}
\definecolor{frenchplum}      {RGB}{129,20,83}


%% Define an Arduino style fore use later %%
\lstdefinestyle{myArduino}{
  language=Arduino,
    %% Add other words needing highlighting below %%
    morekeywords=[1]{},                  % [1] -> dark green
    morekeywords=[2]{FILE_WRITE},        % [2] -> light blue
    morekeywords=[3]{SD, File},          % [3] -> bold orange
    morekeywords=[4]{open, exists, write, SoftwareSerial},      % [4] -> orange
    frame=tb,    
    inputencoding=utf8,
    extendedchars=true,
    literate={è}{{\`{e}}}{1},
    breaklines=true,  
}

\lstdefinestyle{mycpp}{
    language=C++,
    inputencoding=utf8,
    extendedchars=true,
    literate={è}{{\`{e}}}{1},
    %escapeinside={(*******}{*******)}
    escapechar=\£,
    %escapeinside=~~,
    frame=tb,
    tabsize=2,
    mathescape=false,
    breaklines=true,                    % wordwrapping
    postbreak=\mbox{\textcolor{red}{$\hookrightarrow$}\space},         
    basicstyle=\fontsize{9}{11}\ttfamily,
    backgroundcolor=\color{light-gray},
    xleftmargin=.25in,
    showstringspaces=false,
    numbers=left,                    
    numbersep=5pt,                   
    %numberstyle=\color{arduinoGrey},    
    %stepnumber=2, 
    %upquote=true,
    commentstyle=\color{commentgreen},
    keywordstyle=\color{eminence},
    stringstyle=\color{red},
    basicstyle=\small\ttfamily, % basic font setting
    emph={int,char,double,float,unsigned,void,bool},
    emphstyle={\color{blue}},
    % keyword highlighting
    classoffset=1, % starting new class
    otherkeywords={>,<,.,;,-,!,=,~},
    morekeywords={>,<,.,;,-,!,=,~},
    keywordstyle=\color{weborange},
    classoffset=0,
}


\lstdefinestyle{myoutput}
{
    inputencoding=utf8,
    extendedchars=true,
    literate={è}{{\`{e}}}{1},
    tabsize=2,
    frame=tb,
    breaklines=true,                    % wordwrapping
    postbreak=\mbox{\textcolor{red}{$\hookrightarrow$}\space},         
    basicstyle=\fontsize{9}{11}\ttfamily,
    backgroundcolor=\color{light-gray},
    xleftmargin=.25in,
    showstringspaces=false,
    numbers=left,                    
    numbersep=5pt, 
}
%%%%%%%%%%%%%%%%%%%%%%%


\usepackage{siunitx} %pacchetto per le unita' di misura

%%%%%%%%%%%%%%%%%%%%%% per i flowchart
\usepackage{xcolor}
\usepackage{tikz}
\usetikzlibrary{shapes,arrows}
\usetikzlibrary{arrows.meta}
\usetikzlibrary{positioning}
\usetikzlibrary{shapes.geometric}
\usepgflibrary{shapes.symbols}
\usetikzlibrary{shapes.multipart}
\usetikzlibrary{decorations.pathreplacing}

\tikzset{%
  >={Latex[width=2mm,length=2mm]},
  % Specifications for style of nodes:
            rect/.style = {rectangle, rounded corners, draw=black,
                           minimum width=4cm, minimum height=1cm,
                           text centered, font=\sffamily},
           round/.style = {ellipse, draw, draw=black,
                           minimum width=4cm, minimum height=1cm,
                           text centered, font=\sffamily},
       smallrect/.style = {rectangle, rounded corners, draw=black,
                           minimum width=2cm, minimum height=1cm,
                           text centered, font=\sffamily},
 smallrectsplit4/.style = {rectangle split, rectangle split parts=4, 
	                       rectangle split part fill={green!30, none, none, none},
	                       align=center,
	                       rounded corners, draw=black,
                           minimum width=2cm, minimum height=1cm,
                           text centered, font=\sffamily},
        rectpile/.style = {rectangle, outer sep=0pt, align=right,
                           minimum width=1cm, minimum height=0.5cm, font=\sffamily}
}

%\tikzset{%
%    >={Latex[width=2mm,length=2mm]},
%      % Specifications for style of nodes:
%         declare/.style = {trapezium,draw=black, minimum width=4cm, minimum height=1cm, 
%                                trapezium right angle=-70, trapezium left angle=70,
%                                minimum width=4cm, minimum height=1cm,
%                                text centered, font=\sffamily},
%           start/.style = {ellipse, draw, draw=black, minimum width=4cm, 
%                                minimum height=1cm, text centered, font=\sffamily},
%            cond/.style = {diamond, aspect=2, draw, draw=black,
%                                minimum width=4cm, minimum height=1cm,
%                                text centered, font=\sffamily},
%            rect/.style = {rectangle, draw, draw=black,
%                                minimum width=4cm, minimum height=1cm,
%                                text centered, font=\sffamily},
%}
%%%%%%%%%%%%%%%%%%%%%%%%%%%%%%%%%%%%%%%
% 

\pagenumbering{arabic}
\pagestyle{plain}

% per non farlo anadre a capo ovunque 
% va in conflitto con quello che lo fa andare a capo nel codice quindi attenzione <--------
%\usepackage[none]{hyphenat}
\usepackage[italian=nohyphenation,english=nohyphenation]{hyphsubst}

% per togliere gli ident all'inizio dei paragrafi
\setlength{\parindent}{0pt}






\begin{document}

\subsection{programma per l'acquisizione dei dati dal kinect e da arduino}
Questo programma \`e stato usato per acquisire i dati da arduino e dal kinect e scriverli in dei file di testo, ce vengono successivamente elaborati. 
%
%
\subsubsection{librerie usate}
La lista completa delle librerie usate \`e:
\begin{lstlisting}[style=mycpp, caption=librerie usate, captionpos=b]
// libreria usata da visual studio, da togliere in caso si usi un altro ide
#include "pch.h"

// librerie standard per i file stringhe e altro
#include <cstddef>
#include <cstdlib>
#include <fstream>
#include <iomanip>
#include <iostream>
#include <ostream>
#include <string>
#include <math.h>

// librerie create da noi
#include "real_time.h"
#include "kin_file_manager.h"
#include "ard_file_manager.h"
#include "data_structure.h"

// libreria per usare i thread
#include <boost/thread.hpp>

// altre librerie per log, alcuni tipi di dato, un buffer FIFO
#include <boost/log/core.hpp>
#include <boost/call_traits.hpp>
#include <boost/circular_buffer.hpp>
#include <boost/container/vector.hpp>

// altre librerie per il logger
#include <boost/log/attributes.hpp>
#include <boost/log/attributes/scoped_attribute.hpp>
#include <boost/log/expressions.hpp>
#include <boost/log/sinks/sync_frontend.hpp>
#include <boost/log/sinks/text_ostream_backend.hpp>
#include <boost/log/sources/basic_logger.hpp>
#include <boost/log/sources/record_ostream.hpp>
#include <boost/log/sources/severity_logger.hpp>
#include <boost/log/utility/setup/common_attributes.hpp>
#include <boost/log/utility/setup/console.hpp>
#include <boost/log/utility/setup/file.hpp>

#include <boost/smart_ptr/make_shared_object.hpp>
#include <boost/smart_ptr/shared_ptr.hpp>

// altre librerie per i thread 
#include <boost/thread/condition_variable.hpp>
#include <boost/thread/mutex.hpp>
#include <boost/thread/thread.hpp>

// librereria per il tempo
#include <boost/date_time/posix_time/posix_time.hpp>

// libreria fatta da noi per gestire la seriale
#include "Serial_handler.h"

// librerie di Windows e kinect 
#include <Windows.h>
#include <Kinect.h>
#include <Shlobj.h>

// namespace standard
using namespace std;
\end{lstlisting}
%
%
\subsubsection{Librerie custom}
Per alleggerire il file "main" abbiamo spostato delle classi in file esterni, che sono a tutti gli effetti delle librerie a se stanti. Abbiamo costruito librerie per: la gestione del tempo, i file manager, le strutture dati e la gestione della seriale.
%
%
\paragraph{Libreria per misurare il tempo}
In "real\_time.h" ci sono alcune funzioni per prendere il tempo e misurarlo:
\begin{lstlisting}[style=mycpp, caption=librerie usate, captionpos=b]
#pragma once

#ifndef _REAL_TIME_H_
#define _REAL_TIME_H_

#include <boost/chrono.hpp>
#include <boost/timer/timer.hpp>

class real_time
{
private:
  boost::timer::cpu_timer timer;
  boost::timer::cpu_times t;

public:
  /// <summary>
  /// start the time counting
  /// </summary>
  /// <returns></returns>
  void start();

  /// <summary>
  /// return enlapsed time in millisecond
  /// </summary>
  /// <returns></returns>
  float stop();

  /// <summary>
  /// get the time in millisecond
  /// </summary>
  /// <returns></returns>
  uint64_t get_curr_time();

};

#endif // #ifndef _REAL_TIME_H_
\end{lstlisting}
%
%
\paragraph{File manager per il kinect}\label{par:kin_file_mng}
In "kin\_file\_manager.h" ci sono le funzioni per gestire il file di testo relativo al kinect:
\begin{lstlisting}[style=mycpp, caption=librerie usate, captionpos=b]
#pragma once

#ifndef _KIN_FILE_MANAGER_H_
#define _KIN_FILE_MANAGER_H_

#include <iostream>
#include <ostream>
#include <fstream>
#include <string>

// in questa libreria sono dichiarate le varie strutture dati
#include "data_structure.h"

class kin_file_manager
{
private:
  std::fstream f;
  std::ios::_Openmode mode;
  unsigned long int line_id = 0;

public:
  // il costruttore provvede a settare il nome del file e la modalit\`a di apertura
  kin_file_manager(std::string file_name, std::ios::_Openmode mode);

  // questa funzione provvede a scrivere un singolo blocco di dati nel file
  void write_data_line(kinect_data dat);

  // funzione che legge un blocco di dati dal file e lo sposta nella struttura dati
  // se si \`e raggiunta la fine del file la funzione ritorna 0, altrimenti ritorna 1. 
  bool read_data_line(kinect_data* dat);

  // distruttore della classe, chiude il file
  ~kin_file_manager()
  {
  f.close();
  }

  // funzione che chiude il file
  void close()
  {
  f.close();
  }

};

#endif // #ifndef _KIN_FILE_MANAGER_H_
\end{lstlisting}
%
%
\paragraph{File manager per arduino}
La libreria "ard\_file\_manager.h" che contiene le funzioni per la gestione del file di testo relativo ai dati di arduino:
\begin{lstlisting}[style=mycpp, caption=librerie usate, captionpos=b]
#pragma once

#ifndef _ARD_FILE_MANAGER_
#define _ARD_FILE_MANAGER_

#include <iostream>
#include <ostream>
#include <fstream>
#include <string>
#include "data_structure.h"


class ard_file_manager
{
private:
  std::fstream f;
  std::ios::_Openmode mode;
  unsigned long int line_id = 0;

public:
  ard_file_manager(std::string file_name, std::ios::_Openmode mode);

  void write_data_line(arduino_data dat);

  bool read_data_line(arduino_data* dat);

  ~ard_file_manager()
  {
    f.close();
  }

  void close()
  {
    f.close();
  }
};

#endif //#ifndef _ARD_FILE_MANAGER_
\end{lstlisting}
Questa libreria \`e strutturata in modo totalmente simile a quella per la gestione dei file del kinect trattata nel paragrafo \ref{par:kin_file_mng} con l'unica differenza che la struttura dati utilizzata \`e relativa ad arduino e non al kinect.
\\
%
%
\paragraph{strutture dati}
La libreria "data\_structure.h" \`e fondamentale e definisce le strutture dati utilizzate nel resto del programma:
\begin{lstlisting}[style=mycpp, caption=librerie usate, captionpos=b]
#pragma once


#ifndef _DATA_STRUCTURE_H_
#define _DATA_STRUCTURE_H_

#include <iostream>
#include <string>
#include <boost/array.hpp>

// struttura dati per gestire i dati provenienti dal kinect
struct kinect_data
{
  static const int number_of_joints = 3;

  // sottostruttura che contiene i dati di ogni giunto: il suo numero, il giunto attaccato ad esso, la posizione e l'angolo rispetto all'altro giunto
  struct joit_data
  {
    int joint_name = -1;
    int to_joint = -1;
    float position[3] = { 0, 0, 0 };
    float angle[3] = { 0, 0, 0 };
  };

  //Tabelle ad accesso diretto utilizzate per ottimizzare la procedura di estrazione dati dalla
  //classe kinect_class e la conversione dai punti nello spazio dello scheletro ad angoli 
  //del braccio. 
  int needed_joint[number_of_joints] = { 8, 9, 10 };
  int to_joint[24] = { -1, -1, -1, -1, -1, -1, -1, -1, 9, 10, -1, -1, -1, -1, -1, -1, -1, -1, -1, -1, -1, -1, -1, -1 };
  int to_ref_joint[24] = { -1, -1, -1, -1, -1, -1, -1, -1, 0, 1, 2, -1, -1, -1, -1, -1, -1, -1, -1, -1, -1, -1, -1, -1 }; // tabella di accesso (chiave numero joint kinect) -> (restituisce l-indice del numero del giunto in needed_joint)    
  float ref_Ang_x[number_of_joints] = { 0, 0, 0 }; //per  8, 9
  float ref_Ang_y[number_of_joints] = { 0, 0, 0 }; //per 8, 9
  float ref_Ang_z[number_of_joints] = { 180, 180, 0 }; //per 8, 9
  float mul_fctor_x[number_of_joints] = { 0, 0, 0 }; //per 8, 9
  float mul_fctor_y[number_of_joints] = { 1, 1, 0 }; //per 8, 9
  float mul_fctor_z[number_of_joints] = { 1, 1, 0 }; //per 8, 9

  joit_data needed_joints[number_of_joints];

  // tempo esatto in cui i dati sono stati acquisiti
  uint64_t frame_time = 0;

  // contatore usato nella post elaborazione dei dati
  uint64_t contatore = 0;

  // funzione per verificare se il frame acquisito \`e parziale o completo 
  bool full_frame();

  // funzione per stampare i dati memorizzati
  void print_data();

  float jointAngleX(float *P1, float *P2);

  float jointAngleY(float *P1, float *P2);

  float jointAngleZ(float *P1, float *P2);

  void updateAngles();

  // funzione per convertire il numero del giunto in una stringa
  std::string joint_Enum_ToStr(int n, std::string language);
};

// struttura dati per gestire i dati provenienti da arduino
struct arduino_data
{
  // variabili che rappresentano i dati: dell'accelerometro, del giroscopio, del magnetometro e la temperatura
  float acc_xyz[3];
  float gy_xyz[3];
  float magn_xyz[3];
  float temp;

  // tempo esatto di acquisizione dei dati
  uint64_t frame_time = 0;

  // ?
  uint64_t contatore = 0;

  // funzione che permette di stampare il set di dati attualmente immagazzinato a schermo 
  void print_data();
};

// struttura template che gestisce un set di dati del dataset
template<typename type_in, std::size_t N, typename type_out, std::size_t M>
struct dataset_data
{
  boost::array<type_in, N> in;
  boost::array<type_out, M> out;

  // stamapa i dati attualmente immagazzinati
  void print_data();
};

#endif //#ifndef _DATA_STRUCTURE_H_
\end{lstlisting}
%
%
\paragraph{Gestione seriale}
Infine, l'ultima libreria fatta da noi ospita la classe che gestisce la seriale per la comunicazione tramite bluetooth con arduino:
\begin{lstlisting}[style=mycpp, caption=librerie usate, captionpos=b]
#pragma once

#ifndef _SERIAL_HADLER_H_
#define _SERIAL_HADLER_H_

#include <iostream>
#include <string>
#include <boost/asio.hpp> 
#include <boost/bind.hpp>
#include <boost/asio/serial_port.hpp> 


class Serial
{
private:
  // dichiaro la porta seriale che verra inizializzata nel costruttore
  boost::asio::serial_port* port;

  // per ricomporre i float ricevuti uso un union
  union Scomp_float
  {
    float n_float;
    uint32_t n_int;
    uint8_t n_bytes[4];
  };

public:
  // nel costruttore si inizializza la seriale e si effettua la connessione al dispositivo bluetooth
  Serial(std::string com);
  
  // effettua una sincronizzazione iniziale tra arduino ed il pc
  void sinc();

  // riceve un float da arduino
  float receive_float();

  // metodo per inviare un carattere
  void send_char(char ch);

  // metodo per ricevere un carattere
  char receive_char();


  // riceve i dati di acc, gy, magn da arduino
  // il metodo ritorna 
  //     -> 0 se la trasmissione \`e andata a buon fine
  //     -> 1 se \`e fallita
  int receive_data(float* acc_xyz, float* g_xyz, float* magn, float* temp);

  ~Serial()
  {
    port->close();
  }
};

#endif // #ifndef _SERIAL_HADLER_H_
\end{lstlisting}
Per trasmettere il blocco di dati da arduino al pc abbiabo sviluppato un "protocollo" di comunicazione, cio\`e i vari dati sono separati da sei tag di controllo che permettono di verificare se il dato \`e stato trasmesso correttamente o no.

\newsavebox{\mylistingboxcodeard}
\newsavebox{\mylistingboxcodepc}
\begin{figure}[h!]
	\centering
	\begin{lrbox}{\mylistingboxcodeard}%
		\begin{minipage}{.40\linewidth}%
			\centering
			\begin{lstlisting}[style=mycpp]
void send_data(float* acc_xyz, float* g_xyz, float* magn, float temp)
{
  send_char('a');
  for (int i = 0; i < 3; i++)
    send_float(acc_xyz[i]);
  send_char('g');
  for (int i = 0; i < 3; i++)
    send_float(g_xyz[i]);
  send_char('m');
  for (int i = 0; i < 3; i++)
    send_float(magn[i]);
  send_char('t');
  send_float(temp);
  send_char(4);
}
			\end{lstlisting}%
		\end{minipage}%
	\end{lrbox}%
	
	
	\begin{lrbox}{\mylistingboxcodepc}%
		\begin{minipage}{.40\linewidth}%
			\centering
			\begin{lstlisting}[style=mycpp]
int Serial::receive_data(float* acc_xyz, float* g_xyz, float* magn, float* temp)
{
  if (receive_char() != 'a')
    return 1;
  for (int i = 0; i < 3; i++)
    acc_xyz[i] = receive_float();
  if (receive_char() != 'g')
    return 1;
  for (int i = 0; i < 3; i++)
    g_xyz[i] = receive_float();
  if (receive_char() != 'm')
    return 1;
  for (int i = 0; i < 3; i++)
    magn[i] = receive_float();
  if (receive_char() != 't')
    return 1;
  temp[0] = receive_float();
  if (receive_char() != 4)
    return 1;
  return 0;
}
			\end{lstlisting}%
		\end{minipage}%
	\end{lrbox}%
	
	
	\begin{tabular}{@{}cc@{}}
		
		
		\begin{subfigure}[b]{.40\textwidth}
			\centering
			\usebox{\mylistingboxcodeard}
			\vspace{-10pt}
			\caption{codice arduino}
  			\label{fig:codeard}
 		\end{subfigure} 
	 	
	 	\hspace{40pt}
		
		\begin{subfigure}[b]{.40\textwidth}
			\centering
			\usebox{\mylistingboxcodepc}
			\vspace{-10pt}
			\caption{codice che gira sul pc}
  			\label{fig:codepc}
 		\end{subfigure} 
		
	\end{tabular}
	
	\caption{codice per l'invio e la ricezione dei dati necessari}
	\label{fig:send_receive_float}
\end{figure}

TODO \newpage

Nel codice in figura \ref{fig:codeard} troviamo la funzione per inviare i dati da arduino, mentre in figura \ref{fig:codepc} troviamo il codice per ricevere gli stessi dati sul pc. Come si pu\`o vedere viene inviato un carattere di controllo prima di ogni dato, questo carattere serve per controllare l'integrit\`a della comunicazione e deve essere ricevuto dal pc. Se ci\`o non accade significa che nella trasmissione \`e andato perso 1 byte e quindi arduino ed il pc vanno risincronizzati. 


TODO \newpage

%
%
\subsubsection{programma principale dell'acquisizione dati}
Oltre a quste classi/strutture integrate in delle librerie separate ci sono altre classi/altro nel main, infatti abbiamo: "namespace logger", "class bounded\_buffer", "class kinect\_class", "class arduino\_class", "class ard\_handler", "class kin\_handler", "void start\_aquire\_data" e "void acquire\_data".
%
%
\paragraph{Logger}
Per prima cosa \`e bene spiegare il logger, che viene usato nel resto del programma. Questa libreria serve esclusivamente per il debug e ha attributi specifici per lavoare con pi\`u thread contemporaneamente in modo asincrono. Serve fondamentalmente a stampare un testo a schermo, ma una volta settate le impostazioni si pu\`o aggiungere a quasto testo il nome del thread il tempo esatto in cui il messaggio viene stampato ed evitare che 2 thread stampino contemporaneamente nella console (se lo facessero il testo si mischierebbe e non si capirebbe nulla). La documentazione approfondita su come funziona questa libreria si pu\`o trovare \href{https://www.boost.org/doc/libs/1_69_0/libs/log/doc/html/index.html#log.moti}{qui}. Nel nostro caso i settaggi usati sono: un id che identifica il thread, un tag che identifica il nome del trhead, un tag che identifica il livello di severit\`a, il tempo assoluto e la stringa da stampare. Ad esempio se questa riga \`e posta all'inizio del main e il main thread viene chiamato "main":
%
\begin{lstlisting}[style=mycpp, caption=librerie usate, captionpos=b]
BOOST_LOG_SEV(slg, logger::normal) << "hello log";
\end{lstlisting}
%
Allora viene stampato a schermo un messaggio formattato in questo modo:
%
\begin{lstlisting}[style=myoutput, caption=librerie usate, captionpos=b]
id->(0x24a8): main < normal >   time->[00:00:00.000404]  -> hello log
\end{lstlisting}
%
Ci\`o consente di debuggare facilmente operazioni asincrone su pi\`u trhead.
Il codice relativo hai settaggi necessari per ottenere quel tipo di formattazione \`e il "namespace logger":
\begin{lstlisting}[style=mycpp, caption=librerie usate, captionpos=b]
namespace logger
{
  // definizione dei divesti livelli di severit\`a
  enum severity_level
  {
    normal,
    warning,
    error,
    critical
  };
  
  // definizione degli attributi usati
  BOOST_LOG_ATTRIBUTE_KEYWORD(severity, "Severity", severity_level)
  BOOST_LOG_ATTRIBUTE_KEYWORD(tag_attr, "Tag", std::string)
  BOOST_LOG_ATTRIBUTE_KEYWORD(timeline, "Timeline", boost::log::attributes::timer::value_type)
  BOOST_LOG_ATTRIBUTE_KEYWORD(tread_id, "Tread_id", boost::thread::id)
  BOOST_LOG_ATTRIBUTE_KEYWORD(tread_name, "tread_name", std::string)

  // provvede a sostituire il numero del severity_level con la relativa stringa
  std::ostream& operator<< (std::ostream& strm, severity_level level){...}

  // inizializzazione della formattazione del logger
  void init(){...}

  // definizione dei vari parametri
  boost::log::sources::severity_logger< severity_level > f_init(){...}
  void attr_thread(...){...}
  void attr_time(...){...}
  void attr_tag(...){...}
  void attr_thread_name(...){...}
};
\end{lstlisting}
Inoltre per funzionare correttamente richiede che all'inizio del main siano posti alcuni settaggi:
\begin{lstlisting}[style=mycpp, caption=librerie usate, captionpos=b]
// inizializzazione logger
logger::init();
auto slg = logger::f_init();
attr_thread(&slg);
attr_time(&slg);

// imposto il nome del main trhead in "main"
logger::attr_thread_name(&slg, "main");
\end{lstlisting}
%
%
\paragraph{Buffer FIFO}
Un altra classe usata nel resto del programma \`e "class bounded\_buffer", questa classe ospita funzioni specifiche per immagazzinare temporaneamente dei dati gestendone l'ingresso e l'uscita dal buffer. Fondamentalmente costituisce un registro FIFO indipendente dal tipo di dato che si utilizza. Questo tipo di buffer viene comunemente usato in caso si ha la struttura thread che produce dati - thread che consuma dati: il produttore acquisisce i dati e li inserisce nel buffer, il consumatore estrae i dati e li processa. La dimensione del buffer viene specificata quando lo si inizializza e deve essere sufficentemente grande in modo da garantire la non perdita di dati e il non rallentamento del trhead produttore. Nel nostro caso abbiamo adottato questo tipo di buffer per poter garantire l'asincronicit\`a delle acquisizioni tra arduino e il kinect, cos\`i facendo si garantisce la massima velocit\`a tra le acquisizioni dei dati. Il codice relativo a questa classe \`e stato quasi interamente presto dagli esempi della libreria boost, \href{https://www.boost.org/doc/libs/1_69_0/doc/html/circular_buffer/examples.html}{qui} ed \`e:
\begin{lstlisting}[style=mycpp, caption=librerie usate, captionpos=b, label={lst:FIFO}]
template <class T>
class bounded_buffer
{
public:

  typedef boost::circular_buffer<T> container_type;
  typedef typename container_type::size_type size_type;
  typedef typename container_type::value_type value_type;
  typedef typename boost::call_traits<value_type>::param_type param_type;

  // nel costruttore si imposta la capacit\`a del buffer
  explicit bounded_buffer(size_type capacity) : m_unread(0), m_container(capacity) {}

  // questa funzione permette di aggiungere un elemeto al buffer
  void push_front(param_type item){...}

  // questa funzione permette di estrarre un elemento dal buffer 
  void pop_back(value_type* pItem){...}

  // restituisce la percentuale di riempimento del buffer
  float percentage_of_filling(){...}

  // svuota il buffer
  void flush(){...}

private:
  bounded_buffer(const bounded_buffer&);              // Disabled copy constructor
  bounded_buffer& operator = (const bounded_buffer&); // Disabled assign operator

  bool is_not_empty() const { return m_unread > 0; }

  bool is_not_full() const { return m_unread < m_container.capacity(); }

  size_type m_unread;
  container_type m_container;
  boost::mutex m_mutex;
  boost::condition_variable m_not_empty;
  boost::condition_variable m_not_full;
};
\end{lstlisting}


\begin{figure}[h!]
\centering
\begin{tikzpicture}[>=latex,font=\sffamily,semithick,scale=2.25]
    \fill [green!25] (0,0) -- (67.5:1) arc [end angle=-22.5, start angle=67.5, radius=1] -- cycle;
    \draw [thick] (0,0) circle (1);
    \foreach \angle in {90,67.5,...,-67.5}
        \draw (\angle:1) -- (\angle-180:1);
    \node [circle,thick,fill=white,draw=black,align=center,minimum size=3cm] at (0,0) {FIFO implementato\\ con un buffer circolare};
    \draw [<-] (56.25:1) -- (56.25:1.25) -- +(.333,0)
        node [right,inner xsep=.333cm] (Head) {Head (extract)};
    \draw [<-] (-33.75:1) -- (-33.75:1.25) -- +(.333,0)
        node [right,inner xsep=.333cm] (Tail) {Tail (insert)};
    \draw [->,shorten >=5pt,shorten <=5pt] (Tail.west) to [bend right] 
        node [midway,sloped,above,allow upside down] {\footnotesize 4 dati nel FIFO}
    (Head.west);
\end{tikzpicture}
\label{fig:FIFO}
\caption{Schema di funzionamento circular buffer}
\end{figure}
\newpage TODO \\
La classe nel listato \ref{lst:FIFO} utilizza un buffer circolare per implementare il registro FIFO, il funzionamento di questo buffer \`e descritto dalla figura \ref{fig:FIFO}: ogni volta che si aggiunge un dato viene occupata una locazione in testa al buffer ed ogni volta che si rimuove un dato viene liberata una locazione in coda, cio\`o permette di riusare continuamente le stesse locazioni di memoria.
%
%
\paragraph{Classe gestione kinect}
La classe "kinect\_class" provvede a leggere i dati dal kinect ed aggiungerli al suo corrispettivo registro FIFO. L'estrazione dei dati dal kinect \`e stata realizzata sulla base dell'esempio body basic D2D integrato nell'\href{https://www.microsoft.com/en-us/download/details.aspx?id=44561}{sdk browser} fornito da microsoft. Il codice relativo a questa classe \`e:
\begin{lstlisting}[style=mycpp, caption=librerie usate, captionpos=b]
class kinect_class
{
public:

  // Current Kinect
  IKinectSensor* m_pKinectSensor;
  ICoordinateMapper* m_pCoordinateMapper;

  // Body reader
  IBodyFrameReader* m_pBodyFrameReader;

  real_time t;

  bounded_buffer<kinect_data>* FIFO_kin;

  kinect_class(bounded_buffer<kinect_data>* data_FIFO){...}

  void start()
  {
    
    bool flag = false;
    while (true)
    {
      {...}
      
      // prendo il tempo
      uint64_t t1 = t.get_curr_time();  
      
      // acquisisco i dati e li sposto in data
      flag = Update(&data);

      // se l'acquisizione \`e andata a buon fine procedo
      if (flag)
      {
        // prendo un altra volta il tempo
        uint64_t t2 = t.get_curr_time();
        // cos\`i da poter impostare che il tempo in cui sono stati acquisiti i dati sia 
        // la media tra il tempo prima di cominciare l'acquisizione ed il tempo dopo che essa \`e finita
        data.frame_time = (t1 + t2) / 2;

        // inserisco i dati cos\`i acquisiti nel registro fifo
        FIFO_kin->push_front(data);
        
        // faccio aspettare un po di tempo a questo trhead perch\`e altrimenti il kinect da errore, 
        // dato che non supporta un acquisizione a pi\`u di 30fps
        boost::this_thread::sleep(boost::posix_time::milliseconds(35));  al kinect scoppia

      }

      {...}

    }
  }

  // inizializza il kinect, ritorna 1 se ha avuto successo, 0 altrimenti
  int InitializeDefaultSensor(){...}

  // acquisisce i dati e li passa alla funzione per processarli
  bool Update(kinect_data* data)
  {
    {...}
    
    flag = ProcessBody(BODY_COUNT, ppBodies, data);
    
    {...}
    
    return flag;
    
  }

  // questa funzione provvede a spostare i dati da joints a data sistemandoli nella loro struttura dati, 
  // una spiegazione di come ci\`o viene fatto verra data nel capitolo successivo
  bool handle_data(Joint* joints, kinect_data* data){...}
  

  // quetsta funzione separa le diverse persone rilevate e trasferisce le informazioni 
  // relative hai giunti della prima persona alla funzione handle_data
  bool ProcessBody(int nBodyCount, IBody** ppBodies, kinect_data* data)
  {
    {...}
    
    for (int i = 0; i < nBodyCount; ++i)
    {
      {...}
      
      Joint joints[JointType_Count];

      hr = pBody->GetJoints(_countof(joints), joints);
      
      return handle_data(joints, data);
          
      {...}
      
    }
    
    {...}
  }

  // provvede a spegnere il kinect e deallocare le sue risorse nel caso la classe venga distrutta
  ~kinect_class(){...}
  template<class Interface>
  inline void SafeRelease(Interface *& pInterfaceToRelease){...}

};
\end{lstlisting}
%
%
\paragraph{Classe gestione arduino}
La classe "arduino\_class" provvede a caricare continuamente i dati provenienti dal kineck nel registro fifo, infatti la funzione "start" viene lanciata come thread e continua a caricare i dati fino al termine del programma:
\begin{lstlisting}[style=mycpp, caption=librerie usate, captionpos=b]
class arduino_class
{
public:
  Serial* ser;
  real_time t;
  bounded_buffer<arduino_data>* FIFO_ard;

  // il costruttore provvede ad inizializzare la seriale e sincronizzare i 2 dispositivi
  arduino_class(bounded_buffer<arduino_data>* data_FIFO, string com_port)
  {
    ser = new Serial(com_port);
    cout << "serial ok" << endl;
    FIFO_ard = data_FIFO;
    ser->sinc();
  }

  // questa funzione viene lanciata all'avvio del thread e provvede ad acquisire i dati ed inserirli nel registro fifo
  void start()
  {
    while (true)
    {
      arduino_data data;
      
      uint64_t t1 = t.get_curr_time();
      ser->receive_data(data.acc_xyz, data.gy_xyz, data.magn_xyz, &data.temp);
      uint64_t t2 = t.get_curr_time();

	  // per ottenere un accurata misurazione del tempo viene preso prima e dopo l'acquisizione e poi fatta la media dei valori ottenuti
      data.frame_time = (t1 + t2) / 2;

      FIFO_ard->push_front(data);
    }
  }
};
\end{lstlisting}
%
%
\paragraph{Gestore di arduino e del kinect}
Dopo i 2 thread producer si hanno i 2 thread consumer: "class ard\_handler" e "class kin\_handler", che si occupano rispettivamente di salvare su file di testo le informazioni di arduino e del kinect. 
\begin{lstlisting}[style=mycpp, caption=librerie usate, captionpos=b]
/// <summary>
/// handle the data from arduino and write it on a file
/// </summary>
class ard_handler
{
private:
  bounded_buffer<arduino_data>* ard;
  ard_file_manager* f;

public:
  // il costruttore provvede ad inizializzare la classe per gestire i file di testo di arduino
  ard_handler(bounded_buffer<arduino_data>* ard, string file_name)
  {
    this->ard = ard;
    f = new ard_file_manager(file_name, std::ios::out);
  }

  // questa funzione \`e quella che viene effettivamente lanciata all'avvio del thread e provvede a scaricare il buffer scrivendo i dati sul file fino al raggiungimento della quota predichiarata "tot_esempi"
  void start()
  {
	// inizializzazione logger
    auto slg = logger::f_init();
    attr_thread(&slg);
    logger::attr_thread_name(&slg, "ard_handler");

    BOOST_LOG_SEV(slg, logger::normal) << "hello log";

    for (int i = 0; i < tot_esempi; i++)
    {
      arduino_data data_ard;
      ard->pop_back(&data_ard);

      f->write_data_line(data_ard);

    }
    
	// segnalo sulla console qunado il thread ha finito il suo lavoro
    BOOST_LOG_SEV(slg, logger::normal) << "end";
  }
};
\end{lstlisting}
%
%
La classe "kin\_handler" esegue esattamente le stesse operazioni della precedente con la sola differenza che usa "kin\_file\_manager" al posto di "ard\_file\_manager".
\begin{lstlisting}[style=mycpp, caption=librerie usate, captionpos=b]
/// <summary>
/// handle the data from the kinect and write it in the file
/// </summary>
class kin_handler
{
private:
  bounded_buffer<kinect_data>* kin;
  kin_file_manager* f;

public:
  kin_handler(bounded_buffer<kinect_data>* kin, string file_name){...}

  void start(){...}
};
\end{lstlisting}
%
%
\paragraph{Funzioni di lancio dei thead per l'acquisizione ed il salvataggio dei dati}
Dopo tutte queste classi si hanno 2 funzioni che si occupano di inizializzare le altre classi, lanciare tutti i thread neccessari ed acquisire dall'utente i nomi dei file etc.
\begin{lstlisting}[style=mycpp, caption=librerie usate, captionpos=b]
void start_aquire_data(string out_ard, string out_kin, string com_port)
{
  // in inizializzo il logger per questa funzione
  auto slg = logger::f_init();
  attr_thread(&slg);
  logger::attr_thread_name(&slg, "main");

  // dichiaro i 2 buffer FIFO
  bounded_buffer<arduino_data> data_FIFO_ard(20);
  bounded_buffer<kinect_data> data_FIFO_kin(20);

  // chiamo i costruttori delle classi che gestiscono il kinect e arduino
  // al costruttore di arduino gli devo passare la com a cui il bluetooth \`e collegato
  arduino_class a(&data_FIFO_ard, com_port);
  kinect_class k(&data_FIFO_kin);

  // chiamo i costruttori delle classi per la scrittura nei file di testo
  ard_handler a_h(&data_FIFO_ard, out_ard);
  kin_handler k_h(&data_FIFO_kin, out_kin);

  BOOST_LOG_SEV(slg, logger::normal) << "costructor end" << std::endl;

  // lancio i thread che "producono i dati"
  boost::thread ard([&a]() { a.start(); });
  boost::thread kin([&k]() { k.start(); });

  BOOST_LOG_SEV(slg, logger::normal) << "producer thread launced" << std::endl;

  // lancio i thread che "consumnano dati"
  boost::thread ard_h([&a_h]() { a_h.start(); });
  boost::thread kin_h([&k_h]() { k_h.start(); });

  BOOST_LOG_SEV(slg, logger::normal) << "consumer thread launched" << std::endl;

  // aspetto all'infinito (finch\`e il programma non viene chiuso
  kin.join();
  ard.join();
  kin_h.join();
  ard_h.join();
}
\end{lstlisting}
%
%
infine c'\`e la funzione che si occupa di acquisire i dati dall'utente e passarli alla funzione appena vista.
\begin{lstlisting}[style=mycpp, caption=librerie usate, captionpos=b]
void acquire_data()
{
  string ard_file_mane = "";
  string kin_file_mane = "";
  string arduino_com_port = "";

  // acquisisco il nome dei 2 file che conterranno i dati di arduino e i dati del kinect
  cout << "insert the file names: " << endl << "\t arduino -> ";
  std::getline(std::cin, ard_file_mane);
  cout << "\t kinekt -> ";
  std::getline(std::cin, kin_file_mane);

  // acquisisco il numero della com a cui \`e collegato il bluetooth del computer 
  cout << "insert the arduino com port -> : ";
  std::getline(std::cin, arduino_com_port);

  // lancio la funzione passandogli i dati appena acquisiti
  start_aquire_data(ard_file_mane, kin_file_mane, arduino_com_port);
}
\end{lstlisting}
%
%
\paragraph{Main}
Infine il main chiama semplicemente la funzione appena trattata.
\begin{lstlisting}[style=mycpp, caption=librerie usate, captionpos=b]
int main()
{
  logger::init();

  auto slg = logger::f_init();
  attr_thread(&slg);
  attr_time(&slg);
  logger::attr_thread_name(&slg, "main");

  BOOST_LOG_SEV(slg, logger::normal) << "hello log";

  acquire_data();

  return 0;
}
\end{lstlisting}

\subsubsection{Riassunto}
Riassumendo la struttura generale di questo programma si ha:
\begin{figure}[h]
\centering
\begin{tikzpicture}[node distance=2cm, every node/.style={font=\sffamily}, align=center]
%	\draw[help lines] (-10,-10) grid (1,1);
% ora vanno specificate le posizioni 
%	\node (nome_blocco) 	[tipo_blocco, parentela] 		
%								{testo_blocco};
	\node (main)			    [rect, fill=yellow!30] 							
								    {main};
	\node (aquire_data)	        [rect, below of=main, fill=yellow!30] 
								    {aquire\_data};
	\node (start_aquire_data)	[rect, below of=aquire_data, fill=yellow!30]	
								    {start\_aquire\_data};
	\coordinate[below=of start_aquire_data, yshift=-1cm] (c);
	\node (ard_handler)	    [smallrect, left of=c, 
	                            xshift=0.5cm, fill=yellow!30]	
								{ard\_handler};
	%\node (data_FIFO_ard)	[smallrect, left of=ard_handler, 
	%                            xshift=-1.5cm, fill=green!30]	
	%							{data\_FIFO\_ard};	
	\node (data_FIFO_ard) [smallrectsplit4, 
	                            left of=ard_handler, xshift=-1.5cm]	
								{
									data\_FIFO\_ard
									\nodepart{two}
									    dato 1
									\nodepart{three}
									    \vdots
									\nodepart{four}
									    dato n
								};						
	\node (arduino_class)	    [smallrect, left of=data_FIFO_ard, 
	                                xshift=-1.5cm, fill=yellow!30]	
								    {arduino\_class};
	\node (kinect_class)	[smallrect, right of=c, 
	                            xshift=-0.5cm, fill=yellow!30]
								{kinect\_class};
	\node (data_FIFO_kin) [smallrectsplit4, 
	                            right of=kinect_class, xshift=1.5cm]	
								{
									data\_FIFO\_kin
									\nodepart{two}
									    dato 1
									\nodepart{three}
									    \vdots
									\nodepart{four}
									    dato n
								};
	\node (kin_handler)		[smallrect, right of=data_FIFO_kin, 
	                            xshift=1.5cm, fill=yellow!30]
								{kin\_handler};
							
% per le freccie si fa
%	\daraw[tipo_freccia] (origine) "tipo connessione" "eventuale scritta" (destinazione)			
	\draw[->]	(main)	            --                	(aquire_data);
	\draw[->]	(aquire_data)	    --                	(start_aquire_data);
	\draw[->]	(start_aquire_data)	to [in=90,out=-140]	(ard_handler);
	\draw[->]	(start_aquire_data)	to [in=90,out=-140] (arduino_class);
	\draw[->]	(start_aquire_data)	to [in=90,out=-40]	(kin_handler);
	\draw[->]	(start_aquire_data)	to [in=90,out=-40]	(kinect_class);
	%
	\draw[->]	(arduino_class)	    to [in=165,out=0]	(data_FIFO_ard);
	\draw[->]	(data_FIFO_ard)	    to [in=180,out=-35]	(ard_handler);
	\draw[->]	(kinect_class)	    to [in=165,out=0]	(data_FIFO_kin);
	\draw[->]	(data_FIFO_kin)	    to [in=180,out=-35]	(kin_handler);
\end{tikzpicture}
\caption{i programmi usati per ottenere i pesi della rete addestrata (TODO?)}
\end{figure}
\\
Cio\`e il "main" chiama "aquire\_data" che chiama a sua volta "start\_aquire\_data". Dopodich\`e vengono lanciati i 4 thread ("arduino\_class", "ard\_handler", "kinect\_class", "kin\_handler"). I thread che si occupano di prelevare i dati dai dispositivi sono "arduino\_class" e "kinect\_class", che inseriscono i dati nel corrispettivo buffer FIFO. Infine i thread che si occupano di scrivere i dati su i file di testo sono "ard\_handler" e "kin\_handler", che scrivono i dati prelevati dal buffer nel corrispettivo file di testo.





\end{document}































