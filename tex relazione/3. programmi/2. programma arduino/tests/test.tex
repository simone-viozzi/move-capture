\documentclass[10pt,a4paper]{article}

\usepackage[italian]{babel}
\usepackage{amsmath}
\usepackage{amsfonts}
\usepackage{amssymb}

\usepackage[left=1cm,right=1cm,top=1cm,bottom=2cm]{geometry}

\usepackage{txfonts}
\usepackage[T1]{fontenc}
\usepackage[utf8]{inputenc}

\usepackage{titlesec}
\setcounter{secnumdepth}{4}
\titleformat{\paragraph}{\normalfont\normalsize\bfseries}{\theparagraph}{1em}{}
\titlespacing*{\paragraph}{0pt}{3.25ex plus 1ex minus .2ex}{1.5ex plus .2ex}

%per le immagini
\usepackage{graphicx}
\usepackage{subcaption}
\usepackage{wrapfig}
%\usepackage{subfig}

%per i link
\usepackage{hyperref} 

%%%%%%%% per il codice c++
\usepackage{textcomp}
\usepackage{listings}          % for creating language style
\usepackage{listingsutf8}
\input{arduinoLanguage.tex}    % adds the arduino language listing
\definecolor{commentgreen}{RGB}{2,112,10}
\definecolor{eminence}{RGB}{108,48,130}
\definecolor{weborange}{RGB}{255,165,0}
\definecolor{frenchplum}{RGB}{129,20,83}


%% Define an Arduino style fore use later %%
\lstdefinestyle{myArduino}{
  language=Arduino,
    %% Add other words needing highlighting below %%
    morekeywords=[1]{},                  % [1] -> dark green
    morekeywords=[2]{FILE_WRITE},        % [2] -> light blue
    morekeywords=[3]{SD, File},          % [3] -> bold orange
    morekeywords=[4]{open, exists, write, SoftwareSerial},      % [4] -> orange
    frame=tb,    
    inputencoding=utf8,
    extendedchars=true,
    literate={è}{{\`{e}}}{1},
    breaklines=true,  
}

\lstdefinestyle{mycpp}{
    language=C++,
    inputencoding=utf8,
    extendedchars=true,
    literate={è}{{\`{e}}}{1},
    %escapeinside={(*******}{*******)}
    escapechar=\£,
    %escapeinside=~~,
    frame=tb,
    tabsize=2,
    mathescape=false,
    breaklines=true,                    % wordwrapping
    postbreak=\mbox{\textcolor{red}{$\hookrightarrow$}\space},         
    basicstyle=\fontsize{9}{11}\ttfamily,
    backgroundcolor=\color{light-gray},
    xleftmargin=.25in,
    showstringspaces=false,
    numbers=left,                    
    numbersep=5pt,                   
    %numberstyle=\color{arduinoGrey},    
    %stepnumber=2, 
    %upquote=true,
    commentstyle=\color{commentgreen},
    keywordstyle=\color{eminence},
    stringstyle=\color{red},
    basicstyle=\small\ttfamily, % basic font setting
    emph={int,char,double,float,unsigned,void,bool},
    emphstyle={\color{blue}},
    % keyword highlighting
    classoffset=1, % starting new class
    otherkeywords={>,<,.,;,-,!,=,~},
    morekeywords={>,<,.,;,-,!,=,~},
    keywordstyle=\color{weborange},
    classoffset=0,
}


\lstdefinestyle{myoutput}
{
    inputencoding=utf8,
    extendedchars=true,
    literate={è}{{\`{e}}}{1},
    tabsize=2,
    frame=tb,
    breaklines=true,                    % wordwrapping
    postbreak=\mbox{\textcolor{red}{$\hookrightarrow$}\space},         
    basicstyle=\fontsize{9}{11}\ttfamily,
    backgroundcolor=\color{light-gray},
    xleftmargin=.25in,
    showstringspaces=false,
    numbers=left,                    
    numbersep=5pt, 
}
%%%%%%%%%%%%%%%%%%%%%%%


\usepackage{siunitx} %pacchetto per le unita' di misura

%%%%%%%%%%%%%%%%%%%%%% per i flowchart
\usepackage{xcolor}
\usepackage{tikz}
\usetikzlibrary{shapes,arrows}
\usetikzlibrary{arrows.meta}
\usetikzlibrary{positioning}
\usetikzlibrary{shapes.geometric}
\usepgflibrary{shapes.symbols}
\usetikzlibrary{shapes.multipart}
\usetikzlibrary{decorations.pathreplacing}

\tikzset{%
  >={Latex[width=2mm,length=2mm]},
  % Specifications for style of nodes:
            rect/.style = {rectangle, rounded corners, draw=black,
                           minimum width=4cm, minimum height=1cm,
                           text centered, font=\sffamily},
           round/.style = {ellipse, draw, draw=black,
                           minimum width=4cm, minimum height=1cm,
                           text centered, font=\sffamily},
       smallrect/.style = {rectangle, rounded corners, draw=black,
                           minimum width=2cm, minimum height=1cm,
                           text centered, font=\sffamily},
 smallrectsplit4/.style = {rectangle split, rectangle split parts=4, 
	                       rectangle split part fill={green!30, none, none, none},
	                       align=center,
	                       rounded corners, draw=black,
                           minimum width=2cm, minimum height=1cm,
                           text centered, font=\sffamily},
        rectpile/.style = {rectangle, outer sep=0pt, align=right,
                           minimum width=1cm, minimum height=0.5cm, font=\sffamily}
}

%\tikzset{%
%    >={Latex[width=2mm,length=2mm]},
%      % Specifications for style of nodes:
%         declare/.style = {trapezium,draw=black, minimum width=4cm, minimum height=1cm, 
%                                trapezium right angle=-70, trapezium left angle=70,
%                                minimum width=4cm, minimum height=1cm,
%                                text centered, font=\sffamily},
%           start/.style = {ellipse, draw, draw=black, minimum width=4cm, 
%                                minimum height=1cm, text centered, font=\sffamily},
%            cond/.style = {diamond, aspect=2, draw, draw=black,
%                                minimum width=4cm, minimum height=1cm,
%                                text centered, font=\sffamily},
%            rect/.style = {rectangle, draw, draw=black,
%                                minimum width=4cm, minimum height=1cm,
%                                text centered, font=\sffamily},
%}
%%%%%%%%%%%%%%%%%%%%%%%%%%%%%%%%%%%%%%%
% 

\pagenumbering{arabic}
\pagestyle{plain}

% per non farlo anadre a capo ovunque 
% va in conflitto con quello che lo fa andare a capo nel codice quindi attenzione <--------
%\usepackage[none]{hyphenat}


% per togliere gli ident all'inizio dei paragrafi
\setlength{\parindent}{0pt}







\begin{document}



\newsavebox{\mylistingboxone}
\newsavebox{\mylistingboxtwo}

\newsavebox{\mytikzboxone}
\newsavebox{\mytikzboxtwo}


\begin{figure}[h]
	\centering
	\begin{lrbox}{\mylistingboxone}%
		\begin{minipage}{.30\linewidth}%
			\centering
			\begin{lstlisting}[style=mycpp]
struct
{
  int a;
  unsigned char b[4];
}
			\end{lstlisting}%
		\end{minipage}%
	\end{lrbox}%
	
	
	\begin{lrbox}{\mylistingboxtwo}%
		\begin{minipage}{.30\linewidth}%
			\centering
			\begin{lstlisting}[style=mycpp]
union
{
  int a;
  unsigned char b[4];
}
			\end{lstlisting}%
		\end{minipage}%
	\end{lrbox}%
	
	\begin{lrbox}{\mytikzboxone}%
		\begin{minipage}{.50\linewidth}%
			\centering
			\hspace{-1cm}
			\begin{tikzpicture}[every node/.style={rectpile}]
			\node at (0, 0) (A) {0x00};
	        \node [anchor=north] at (A.south) (B) {0x01};
	        \node [anchor=north] at (B.south) (C) {0x02};
	        \node [anchor=north] at (C.south) (D) {0x03};
	        \node [anchor=north] at (D.south) (E) {0x04};
	        \node [anchor=north] at (E.south) (F) {0x05};
	        \node [anchor=north] at (F.south) (G) {0x06};
	        \node [anchor=north] at (G.south) (H) {0x07};
	        
	        \node [anchor=west, draw=black] at (A.east) (M) {};
	        \node [anchor=west, draw=black] at (B.east) (N) {};
	        \node [anchor=west, draw=black] at (C.east) (O) {};
	        \node [anchor=west, draw=black] at (D.east) (P) {};
	        \node [anchor=west, draw=black] at (E.east) (Q) {};
	        \node [anchor=west, draw=black] at (F.east) (R) {};
	        \node [anchor=west, draw=black] at (G.east) (S) {};
	        \node [anchor=west, draw=black] at (H.east) (T) {};
	        
	        \draw [decorate,decoration={brace,amplitude=10pt,raise=0pt},
	            yshift=-0pt,xshift=-0.5cm]
                (2cm,0.25cm) -- (2cm, -1.75cm) node [black,midway,xshift=0.8cm] {\hspace{-8pt}a};
            \draw [decorate,decoration={brace,amplitude=10pt,raise=0pt},
	            yshift=-0pt,xshift=-0.5cm]
                (2cm,-1.75cm) -- (2cm, -3.75) node [black,midway,xshift=0.8cm] {b[ ]};
			\end{tikzpicture}%
		\end{minipage}%
	\end{lrbox}%
	
	
	\begin{lrbox}{\mytikzboxtwo}%
		\begin{minipage}{.50\linewidth}%
			\centering
			\begin{tikzpicture}[every node/.style={rectpile}]
			\node at (0, 0) (A) {0x00};
	        \node [anchor=north] at (A.south) (B) {0x01};
	        \node [anchor=north] at (B.south) (C) {0x02};
	        \node [anchor=north] at (C.south) (D) {0x03};
	        \node [anchor=north] at (D.south) (E) {0x04};
	        \node [anchor=north] at (E.south) (F) {0x05};
	        \node [anchor=north] at (F.south) (G) {0x06};
	        \node [anchor=north] at (G.south) (H) {0x07};
	        
	        \node [anchor=west, draw=black] at (A.east) (M) {};
	        \node [anchor=west, draw=black] at (B.east) (N) {};
	        \node [anchor=west, draw=black] at (C.east) (O) {};
	        \node [anchor=west, draw=black] at (D.east) (P) {};
	        \node [anchor=west, draw=black] at (E.east) (Q) {};
	        \node [anchor=west, draw=black] at (F.east) (R) {};
	        \node [anchor=west, draw=black] at (G.east) (S) {};
	        \node [anchor=west, draw=black] at (H.east) (T) {};
	        
	        \node [draw=black] at (3, 0) (M1) {};
	        \node [anchor=north, draw=black] at (M1.south) (N1) {};
	        \node [anchor=north, draw=black] at (N1.south) (O1) {};
	        \node [anchor=north, draw=black] at (O1.south) (P1) {};
	        \node [anchor=north, draw=black] at (P1.south) (Q1) {};
	        \node [anchor=north, draw=black] at (Q1.south) (R1) {};
	        \node [anchor=north, draw=black] at (R1.south) (S1) {};
	        \node [anchor=north, draw=black] at (S1.south) (T1) {};
	        
	        \draw [decorate,decoration={brace,amplitude=10pt,raise=0pt},
	            yshift=-0pt,xshift=-0.5cm]
                (2cm,0.25cm) -- (2cm, -1.75cm) node [black,midway,xshift=0.8cm] 
                {\hspace{-10pt}a};
            \draw [decorate,decoration={brace,amplitude=10pt,raise=0pt},
	            yshift=-0pt,xshift=-0.5cm]
                (4cm,0.25cm) -- (4cm, -1.75) node [black,midway,xshift=0.8cm] {b[ ]};

            \draw [densely dotted] (1.5,0.25) -- (3,0.25);              
            
            \draw [densely dotted] (1.5,-1.75) -- (3,-1.75);       
            
			\end{tikzpicture}%
		\end{minipage}%
	\end{lrbox}%
	
	\begin{tabular}{@{}cc@{}}
		%
		\begin{tabular}{@{}c@{}}
		
			\begin{subfigure}[b]{.40\textwidth}
				\centering
				\usebox{\mylistingboxone}
				\vspace{-10pt}
				\caption{Another subfigure}
	  			\label{fig:1b}
	  			\vspace{10pt}
	 		\end{subfigure} \\
	 		
	 		\begin{subfigure}[b]{.45\textwidth}
				\centering
				\usebox{\mytikzboxone}
				\vspace{0pt}
				\caption{Another subfigure}
	  			\label{fig:1b}
	  			\vspace{0pt}
	 		\end{subfigure}  
			%
			
		\end{tabular}

		\hspace{-50pt}		
		
		\begin{tabular}{@{}c@{}}
			
			\begin{subfigure}[b]{.40\textwidth}
				\centering
				\usebox{\mylistingboxtwo}
				\vspace{-10pt}
				\caption{Another subfigure}
	  			\label{fig:1b}
	  			\vspace{10pt}
	 		\end{subfigure} \\
			
			
			\begin{subfigure}[b]{.45\textwidth}
				\centering
				\usebox{\mytikzboxtwo}
				\vspace{0pt}
				\caption{Another subfigure}
	  			\label{fig:1b}
	  			\vspace{0pt}
	 		\end{subfigure} 
	 		
		\end{tabular}
		
	\end{tabular}
	
	\caption{A figure}
	\label{fig:1}
\end{figure}


\begin{tikzpicture}[>=latex,font=\sffamily,every node/.style={minimum width=1cm,minimum height=1.5em,outer sep=0pt,draw=black,fill=blue!40}]
        \node at (0,0) (A) {};
        \node [anchor=west, align=right] at (A.east) (B) {dio cane \\ porco};
        \node [anchor=west, fill=white!50] at (B.east) (C) {1};
        \node [anchor=west] at (C.east) (D) {2};
        \node [anchor=west] at (D.east) (E) {3};
        \node [anchor=west] at (E.east) (F) {};
        \node [anchor=west] at (F.east) (G) {};
        \draw [->,shorten >=2pt,shorten <=2pt,semithick] (G.south) -- +(0,-1em) -| (A);
\end{tikzpicture}


\end{document}