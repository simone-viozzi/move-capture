\documentclass[10pt,a4paper]{article}

\usepackage[italian]{babel}
\usepackage{amsmath}
\usepackage{amsfonts}
\usepackage{amssymb}

\usepackage[left=1cm,right=1cm,top=1cm,bottom=2cm]{geometry}

\usepackage{txfonts}
\usepackage[T1]{fontenc}
\usepackage[utf8]{inputenc}

\usepackage{titlesec}
\setcounter{secnumdepth}{4}
\titleformat{\paragraph}{\normalfont\normalsize\bfseries}{\theparagraph}{1em}{}
\titlespacing*{\paragraph}{0pt}{3.25ex plus 1ex minus .2ex}{1.5ex plus .2ex}

%per le immagini
\usepackage{graphicx}
\usepackage{subcaption}
\usepackage{wrapfig}

%per i link
\usepackage{hyperref} 

%%%%%%%% per il codice c++
\usepackage{textcomp}
\usepackage{listings}          % for creating language style
\usepackage{listingsutf8}
\input{arduinoLanguage.tex}    % adds the arduino language listing
\definecolor{commentgreen}{RGB}{2,112,10}
\definecolor{eminence}{RGB}{108,48,130}
\definecolor{weborange}{RGB}{255,165,0}
\definecolor{frenchplum}{RGB}{129,20,83}


%% Define an Arduino style fore use later %%
\lstdefinestyle{myArduino}{
  language=Arduino,
    %% Add other words needing highlighting below %%
    morekeywords=[1]{},                  % [1] -> dark green
    morekeywords=[2]{FILE_WRITE},        % [2] -> light blue
    morekeywords=[3]{SD, File},          % [3] -> bold orange
    morekeywords=[4]{open, exists, write, SoftwareSerial},      % [4] -> orange
    frame=tb,    
    inputencoding=utf8,
    extendedchars=true,
    literate={è}{{\`{e}}}{1},
    breaklines=true,  
}

\lstdefinestyle{mycpp}{
    language=C++,
    inputencoding=utf8,
    extendedchars=true,
    literate={è}{{\`{e}}}{1},
    %escapeinside={(*******}{*******)}
    escapechar=\£,
    %escapeinside=~~,
    frame=tb,
    tabsize=2,
    mathescape=false,
    breaklines=true,                    % wordwrapping
    postbreak=\mbox{\textcolor{red}{$\hookrightarrow$}\space},         
    basicstyle=\fontsize{9}{11}\ttfamily,
    backgroundcolor=\color{light-gray},
    xleftmargin=.25in,
    showstringspaces=false,
    numbers=left,                    
    numbersep=5pt,                   
    %numberstyle=\color{arduinoGrey},    
    %stepnumber=2, 
    %upquote=true,
    commentstyle=\color{commentgreen},
    keywordstyle=\color{eminence},
    stringstyle=\color{red},
    basicstyle=\small\ttfamily, % basic font setting
    emph={int,char,double,float,unsigned,void,bool},
    emphstyle={\color{blue}},
    % keyword highlighting
    classoffset=1, % starting new class
    otherkeywords={>,<,.,;,-,!,=,~},
    morekeywords={>,<,.,;,-,!,=,~},
    keywordstyle=\color{weborange},
    classoffset=0,
}

\lstdefinestyle{mycuda}{
    language=C++,
    inputencoding=utf8,
    extendedchars=true,
     literate=
    {á}{{\'a}}1 {é}{{\'e}}1 {í}{{\'i}}1 {ó}{{\'o}}1 {ú}{{\'u}}1
    {Á}{{\'A}}1 {É}{{\'E}}1 {Í}{{\'I}}1 {Ó}{{\'O}}1 {Ú}{{\'U}}1
    {à}{{\`a}}1 {è}{{\`e}}1 {ì}{{\`i}}1 {ò}{{\`o}}1 {ù}{{\`u}}1
    {À}{{\`A}}1 {È}{{\'E}}1 {Ì}{{\`I}}1 {Ò}{{\`O}}1 {Ù}{{\`U}}1
    {ä}{{\"a}}1 {ë}{{\"e}}1 {ï}{{\"i}}1 {ö}{{\"o}}1 {ü}{{\"u}}1
    {Ä}{{\"A}}1 {Ë}{{\"E}}1 {Ï}{{\"I}}1 {Ö}{{\"O}}1 {Ü}{{\"U}}1
    {â}{{\^a}}1 {ê}{{\^e}}1 {î}{{\^i}}1 {ô}{{\^o}}1 {û}{{\^u}}1
    {Â}{{\^A}}1 {Ê}{{\^E}}1 {Î}{{\^I}}1 {Ô}{{\^O}}1 {Û}{{\^U}}1
    {œ}{{\oe}}1 {Œ}{{\OE}}1 {æ}{{\ae}}1 {Æ}{{\AE}}1 {ß}{{\ss}}1
    {ç}{{\c c}}1 {Ç}{{\c C}}1 {ø}{{\o}}1 {å}{{\r a}}1 {Å}{{\r A}}1
    {€}{{\EUR}}1 {£}{{\pounds}}1,
    %escapeinside={(*******}{*******)}
    escapechar=\£,
    %escapeinside=~~,
    frame=tb,
    tabsize=2,
    mathescape=false,
    breaklines=true,                    % wordwrapping
    postbreak=\mbox{\textcolor{red}{$\hookrightarrow$}\space},         
    basicstyle=\fontsize{9}{11}\ttfamily,
    backgroundcolor=\color{light-gray},
    xleftmargin=.25in,
    showstringspaces=false,
    numbers=left,                    
    numbersep=5pt,                   
    %numberstyle=\color{arduinoGrey},    
    %stepnumber=2, 
    %upquote=true,
    commentstyle=\color{commentgreen},
    keywordstyle=\color{eminence},
    stringstyle=\color{red},
    basicstyle=\small\ttfamily, % basic font setting
    emph={int,char,double,float,unsigned,void,bool},
    emphstyle={\color{blue}},
    morekeywords = [2]{cudaMalloc, cudaFree,
        __global__, __shared__, __device__, __host__,
        __syncthreads},
    keywordstyle=[2]\color{magenta},
    % keyword highlighting
    classoffset=1, % starting new class
    otherkeywords={>,<,.,;,-,!,=,~},
    morekeywords=[3]{>,<,.,;,-,!,=,~zz},
    keywordstyle=[3]\color{weborange},
    classoffset=0,
}

\lstdefinestyle{mycsharp}{
    language=[Sharp]C,
    inputencoding=utf8,
    extendedchars=true,
     literate=
    {á}{{\'a}}1 {é}{{\'e}}1 {í}{{\'i}}1 {ó}{{\'o}}1 {ú}{{\'u}}1
    {Á}{{\'A}}1 {É}{{\'E}}1 {Í}{{\'I}}1 {Ó}{{\'O}}1 {Ú}{{\'U}}1
    {à}{{\`a}}1 {è}{{\`e}}1 {ì}{{\`i}}1 {ò}{{\`o}}1 {ù}{{\`u}}1
    {À}{{\`A}}1 {È}{{\'E}}1 {Ì}{{\`I}}1 {Ò}{{\`O}}1 {Ù}{{\`U}}1
    {ä}{{\"a}}1 {ë}{{\"e}}1 {ï}{{\"i}}1 {ö}{{\"o}}1 {ü}{{\"u}}1
    {Ä}{{\"A}}1 {Ë}{{\"E}}1 {Ï}{{\"I}}1 {Ö}{{\"O}}1 {Ü}{{\"U}}1
    {â}{{\^a}}1 {ê}{{\^e}}1 {î}{{\^i}}1 {ô}{{\^o}}1 {û}{{\^u}}1
    {Â}{{\^A}}1 {Ê}{{\^E}}1 {Î}{{\^I}}1 {Ô}{{\^O}}1 {Û}{{\^U}}1
    {œ}{{\oe}}1 {Œ}{{\OE}}1 {æ}{{\ae}}1 {Æ}{{\AE}}1 {ß}{{\ss}}1
    {ç}{{\c c}}1 {Ç}{{\c C}}1 {ø}{{\o}}1 {å}{{\r a}}1 {Å}{{\r A}}1
    {€}{{\EUR}}1 {£}{{\pounds}}1,
    %escapeinside={(*******}{*******)}
    escapechar=\£,
    %escapeinside=~~,
    frame=tb,
    tabsize=2,
    mathescape=false,
    breaklines=true,                    % wordwrapping
    postbreak=\mbox{\textcolor{red}{$\hookrightarrow$}\space},         
    basicstyle=\fontsize{9}{11}\ttfamily,
    backgroundcolor=\color{light-gray},
    xleftmargin=.25in,
    showstringspaces=false,
    numbers=left,                    
    numbersep=5pt,                   
    %numberstyle=\color{arduinoGrey},    
    %stepnumber=2, 
    %upquote=true,
    commentstyle=\color{commentgreen},
    keywordstyle=\color{eminence},
    stringstyle=\color{red},
    basicstyle=\small\ttfamily, % basic font setting
    emph={int,char,double,float,unsigned,void,bool,byte},
    emphstyle={\color{blue}},
    % keyword highlighting
    classoffset=1, % starting new class
    otherkeywords={>,<,.,;,-,!,=,~},
    morekeywords=[3]{>,<,.,;,-,!,=,~zz},
    keywordstyle=[3]\color{weborange},
    classoffset=0,
}

\lstdefinestyle{myoutput}
{
    inputencoding=utf8,
    extendedchars=true,
    literate={è}{{\`{e}}}{1},
    tabsize=2,
    frame=tb,
    breaklines=true,                    % wordwrapping
    postbreak=\mbox{\textcolor{red}{$\hookrightarrow$}\space},         
    basicstyle=\fontsize{9}{11}\ttfamily,
    backgroundcolor=\color{light-gray},
    xleftmargin=.25in,
    showstringspaces=false,
    numbers=left,                    
    numbersep=5pt, 
}
%%%%%%%%%%%%%%%%%%%%%%%


\usepackage{siunitx} %pacchetto per le unita' di misura

%%%%%%%%%%%%%%%%%%%%%% per i flowchart
\usepackage{xcolor}
\usepackage{tikz}
\usetikzlibrary{shapes,arrows}
\usetikzlibrary{arrows.meta}
\usetikzlibrary{positioning}
\usetikzlibrary{shapes.geometric}
\usepgflibrary{shapes.symbols}
\usetikzlibrary{shapes.multipart}

\tikzset{%
  >={Latex[width=2mm,length=2mm]},
  % Specifications for style of nodes:
            rect/.style = {rectangle, rounded corners, draw=black,
                           minimum width=4cm, minimum height=1cm,
                           text centered, font=\sffamily},
           round/.style = {ellipse, draw, draw=black,
                           minimum width=4cm, minimum height=1cm,
                           text centered, font=\sffamily},
       smallrect/.style = {rectangle, rounded corners, draw=black,
                           minimum width=2cm, minimum height=1cm,
                           text centered, font=\sffamily},
 smallrectsplit4/.style = {rectangle split, rectangle split parts=4, 
	                       rectangle split part fill={green!30, none, none, none},
	                       align=center,
	                       rounded corners, draw=black,
                           minimum width=2cm, minimum height=1cm,
                           text centered, font=\sffamily},
}

%\tikzset{%
%    >={Latex[width=2mm,length=2mm]},
%      % Specifications for style of nodes:
%         declare/.style = {trapezium,draw=black, minimum width=4cm, minimum height=1cm, 
%                                trapezium right angle=-70, trapezium left angle=70,
%                                minimum width=4cm, minimum height=1cm,
%                                text centered, font=\sffamily},
%           start/.style = {ellipse, draw, draw=black, minimum width=4cm, 
%                                minimum height=1cm, text centered, font=\sffamily},
%            cond/.style = {diamond, aspect=2, draw, draw=black,
%                                minimum width=4cm, minimum height=1cm,
%                                text centered, font=\sffamily},
%            rect/.style = {rectangle, draw, draw=black,
%                                minimum width=4cm, minimum height=1cm,
%                                text centered, font=\sffamily},
%}
%%%%%%%%%%%%%%%%%%%%%%%%%%%%%%%%%%%%%%%
% 

\pagenumbering{arabic}
\pagestyle{plain}

% per non farlo anadre a capo ovunque 
% va in conflitto con quello che lo fa andare a capo nel codice quindi attenzione <--------
%\usepackage[none]{hyphenat}


% per togliere gli ident all'inizio dei paragrafi
\setlength{\parindent}{0pt}


\begin{document}
\subsection{Interfaccia grafica}
L'interfaccia grafica dell'applicativo è stata realizzata in Unity, un popolare motore grafico per videogiochi e applicazioni android e ios.
E' stato realizzato un rig (un manichino) attraverso blender con uno scheletro interfacciabile con Unity, in modo da poterne controllare i movimenti attraverso l'inserimento degli angoli dei vari "ossi" dello scheletro.
Lo script dell'interfaccia grafica è stato scritto in C\# e si basa su un client che richiede continuamente al server i dati di uscita della rete neurale.


\begin{figure}[h!]
  	\centering
    \includegraphics[height=200pt]{dummy.PNG}
    \caption{rig utilizzato per la renderizzazione}
\end{figure}

\subsubsection{classe syncSocketServer}
Questa parte illustra la classe utilizzata per far effettuare allo script in unity la richieta al server centrale dei dati elaborati.

\paragraph{librerie utilizzate}
tale classe è stata implementata nella parte unity e ha richiesto l'importazione di UnityEngine, oltre alle librerie per la comunicazione socket e per la manipolazione dei byte ricevuti dal main server. 
\begin{lstlisting}[style=mycsharp, caption=librerie class client C\#, captionpos=b]
using System;
using System.Net;
using System.Net.Sockets;
using System.Runtime.InteropServices;
using System.Text;
using System.Collections;
using System.Collections.Generic;
using UnityEngine;

\end{lstlisting}

\paragraph{classe client C\# per la comunicazione}
In questa sezione è illustrato la classe che permette la comunicazione con il main server. 
Si noti come l'utilizzo della classe in unity richiede che essa sia figlia della classe MonoBehaviour, interna di unity.
\begin{lstlisting}[style=mycsharp, caption=class client C\#, captionpos=b]

public class SyncSocketClient : MonoBehaviour {
    //struttura union per la conversione Bytes -> float
    [StructLayout(LayoutKind.Explicit)]
    public struct Ubones
    {
        [FieldOffset(0)] public float fBones;
        [FieldOffset(0)] public byte strBones1;
        [FieldOffset(1)] public byte strBones2;
        [FieldOffset(2)] public byte strBones3;
        [FieldOffset(3)] public byte strBones4;
    }

    //Oggetti socket 
    public IPHostEntry ipHostInfo;
    public IPAddress ipAddress;
    public IPEndPoint remoteEP;
    public Socket sender;
    private const int number_of_joints = 2;
    private const int number_of_float = number_of_joints * 3;


    // alloco la struttura di conversione
    public Ubones[] bones = new Ubones[number_of_float]; //4 joints

    // dichiaro il vettore di buffer
    public byte[] bytes;

    //settaggio parametri e inizzializazione comunicazione
    public SyncSocketClient(int Port){ ... }

    //chiusura comunicazione
    public void CloseConnection(){ ... }

    // funzione per la ricezione e la conversione float
    public void ReciveFloat(){ ... }
	
	//funzione per la ricezione della stringa dal server
    public string ReciveString(){ ... }

    //trasmissione stringa
    public void sendMessage(string msg){ ... }

    //funzione di comunicazione, invia il codice di richiesta, riceve i byte li carica nell' union 
    public void TestComunication(){ ... }
	
	//funzione per la reinizializzazione delle variabili bones
    public void initBones(){ ... }
    }
}
\end{lstlisting}

\subsubsection{classe di controllo del manichino}
In questa sezione viene illustrata la classe interna al motore grafico realizzata per il caricamento dei dati ricevuti dal server nella struttura di dell'oggetto 
\paragraph{librerie della classe controllo}
\begin{lstlisting}[style=mycsharp, caption=Librerie classe controllo C\#, captionpos=b]
using System.Collections.Generic;
using System.IO;
using System.Text;
using UnityEngine; 
\end{lstlisting}
\paragraph{corpo della classe controllo}
Anche questa classe è stata resa figlia di MonoBehaviour, il settaggio delle variabili viene effettuat dal metodo Start() richiamato da Unity all'avvio del programma, in questo caso però l'utilizzo della classe è richiamato attivamente da Unity attraverso il metodo Update() che viene richiamato una volta per ogni aggiornamento dello schermo. 
\begin{lstlisting}[style=mycsharp, caption=class client C\#, captionpos=b]
public class ProvaControllo : MonoBehaviour {
	//dichiarazione della classe socketClient sincrono
	//e dell'oggetto di controllo del manichino
    private SyncSocketClient socket;
    public GameObject[] bones;
    
    private List<string> Tags = new List<string>();
    
    /*{ 4, 6, 3, 5, 8};*/ // indici degli oggetti da aggiornare nella lista GameObject  
    public int[] neededObjects = new int[] {3, 5}; 
    
    //ref kinect{ 4, 5, 8, 9}; // indici degli oggetti da aggiornare nella struct Ubones della com socket
    public int[] neededBonesF = new int[] {0, 1}; 
	
	//vettori da tre per il passaggio degli angoli
    public Vector3 BraccioDx;
    public Vector3 avanBraccioDx;
    public Vector3 BraccioSx;
    public Vector3 avanBraccioSx;

    // Metodo di inizializazzione della classe, richiamato all'avvio
    void Start () {
    
    	//attivazione della comunicazione
        socket = new SyncSocketClient(9999);
        
        //inizializazzione della struttura di conversione byte/float
        socket.initBones();
		
		//creazione della lista dei tag
        Tags.Add("Root"); //0
        Tags.Add("Bacino"); //1
        Tags.Add("schiena"); //2
        Tags.Add("BraccioDx"); //3
        Tags.Add("BraccioSx"); //4
        Tags.Add("AvanbraccioDx"); //5
        Tags.Add("AvanbraccioSx"); //6
        Tags.Add("testa"); //7
        Tags.Add("schiena"); //8
        
		//inizializazzione dell'array di oggetti del gioco
        bones = new GameObject[10];
        
        //link degli oggetti creati alle ossa del manichino
        int i = 0;
        foreach (string Tag in Tags) {
            bones[i] = GameObject.FindGameObjectWithTag(Tag);
            i++;
        }
    }
    
	// Update is called once per frame
	void Update () {
		
		//richiesta al server centrsle dei nuovi dati e aggiornamento 
		//della struttura interna alla classe socket
        socket.TestComunication();
		
		//caricamento dei dati ricevuti sulla struttura unity del manichino
        for (int i = 0; i < 2; i++)
        {   
            bones[neededObjects[i]].transform.rotation = Quaternion.Euler(socket.bones[neededBonesF[i]*3].fBones, socket.bones[(neededBonesF[i] * 3) + 1].fBones, socket.bones[(neededBonesF[i] * 3) + 2].fBones);
        }
}

\end{lstlisting}

\end{document}