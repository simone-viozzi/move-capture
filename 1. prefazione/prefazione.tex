\documentclass[10pt,a4paper]{article}

\usepackage[italian]{babel}
\usepackage{amsmath}
\usepackage{amsfonts}
\usepackage{amssymb}

\usepackage[left=1cm,right=1cm,top=1cm,bottom=2cm]{geometry}

\usepackage{txfonts}
\usepackage[T1]{fontenc}

%\usepackage{blindtext}
%\usepackage[T1]{fontenc}
\usepackage[utf8]{inputenc}

\usepackage{titlesec}
\setcounter{secnumdepth}{4}
\titleformat{\paragraph}{\normalfont\normalsize\bfseries}{\theparagraph}{1em}{}
\titlespacing*{\paragraph}{0pt}{3.25ex plus 1ex minus .2ex}{1.5ex plus .2ex}

\usepackage{graphicx}
\usepackage{subcaption}

\usepackage{wrapfig}

\pagenumbering{arabic}
\pagestyle{plain}
\setlength{\parindent}{0pt}

\begin{document}
 
\section{Prefazione}
 
\subsection{scopo del progetto}
L'obbiettivo di questo progetto \`e la realizzazione di una modello di machine learning in grado di apprendere le relazioni tra i dati fruiti da una scheda embeded dotata di accelerometro, giroscopio e magnetometro, e la posizione spaziale di un braccio umano.
Il problema che ci si \`e posti di risolvere poteva essere risolto in modo deterministico ma per lo scopo del progetto si voleva risolvere il problema tramite un algoritmo di ML, per poter verificare l'efficacia di questo metodo per la risoluzione di un problema con soluzione nota.


\subsection{introduzione generale alle tecniche utilizzate}
Per la realizzazione del progetto è stata realizzata una scheda con microcontrollore arduino, provvista di bluetooth, accelerometro, magnetometro, alimentata a batteria.
\'E stata realizzata una libreria in C++/CUDA per la creazione, l'addestramento e l'utilizzo di una rete neuronale con supporto per CPU e GPU.
\'E stato realizzato un secondo programma C++, per l'acquisizione dei dati della scheda e per l'acquisizione dei punti dello scheletro forniti dal kinect, lo stesso programma si occupa della sincronizzazione di tali dati e della creazione di un dataset necessario per il processo di addestramento della rete neuronale.
In fine è stato realizzato un programma C\# che sfrutta il game engine Unity per la realizzazione dell'ambiente grafico, che rende possibile la visualizzazione dell'output della rete, muovendo un manichino secondo gli input della scheda.
Quest'ultima operazione è stata effettuata mettendo in comunicazione il programma C++ che si occuppa dell'acquisizione dei dati della scheda, e che ne esegue l'input nel modello addestrato, per poi passare i dati elaborati tramite comunicazione socket interna al motore grafico.       


\end{document}