\documentclass[10pt,a4paper]{article}

\usepackage[italian]{babel}
\usepackage{amsmath}
\usepackage{amsfonts}
\usepackage{amssymb}

\usepackage[left=1cm,right=1cm,top=1cm,bottom=2cm]{geometry}

\usepackage{txfonts}
\usepackage[T1]{fontenc}

%\usepackage{blindtext}
%\usepackage[T1]{fontenc}
\usepackage[utf8]{inputenc}

\usepackage{titlesec}
\setcounter{secnumdepth}{4}
\titleformat{\paragraph}{\normalfont\normalsize\bfseries}{\theparagraph}{1em}{}
\titlespacing*{\paragraph}{0pt}{3.25ex plus 1ex minus .2ex}{1.5ex plus .2ex}

\usepackage{graphicx}
\usepackage{subcaption}

\usepackage{wrapfig}

\pagenumbering{arabic}
\pagestyle{plain}
\setlength{\parindent}{0pt}

\begin{document}
 
\section{Prefazione}
(..boh!)
 
\subsection{scopo del progetto}
L'obbiettivo di questo progetto \`e la realizzazione di una modello di machine learning in grado di apprendere le relazioni tra i dati fruiti da una scheda embeded dotata di accelerometro, giroscopio e magnetometro, e la posizione spaziale di un braccio umano.
Il problema che ci si \`e posti di risolvere poteva essere risolto in modo deterministico ma per lo scopo del progetto si voleva risolvere il problema tramite un algoritmo di ML, per poter verificare l'efficacia di questo metodo per la risoluzione di un problema con soluzione nota.


\subsection{introduzione generale alle tecniche utilizzate}
Per questo progetto abbiamo utilizzato 

\end{document}