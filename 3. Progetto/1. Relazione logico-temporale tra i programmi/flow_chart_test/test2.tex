\documentclass[10pt,a4paper]{article}

\usepackage[italian]{babel}
\usepackage{amsmath}
\usepackage{amsfonts}
\usepackage{amssymb}

\usepackage[left=1cm,right=1cm,top=1cm,bottom=2cm]{geometry}

\usepackage{txfonts}
\usepackage[T1]{fontenc}
\usepackage[utf8]{inputenc}

\usepackage{titlesec}
\setcounter{secnumdepth}{4}
\titleformat{\paragraph}{\normalfont\normalsize\bfseries}{\theparagraph}{1em}{}
\titlespacing*{\paragraph}{0pt}{3.25ex plus 1ex minus .2ex}{1.5ex plus .2ex}

\usepackage{graphicx}
\usepackage{subcaption}

\usepackage{wrapfig}

%%%%%%%% per il codice c++
\usepackage{listings}          % for creating language style
\input{arduinoLanguage.tex}    % adds the arduino language listing

%% Define an Arduino style fore use later %%
\lstdefinestyle{myArduino}{
  language=Arduino,
  %% Add other words needing highlighting below %%
  morekeywords=[1]{},                  % [1] -> dark green
  morekeywords=[2]{FILE_WRITE},        % [2] -> light blue
  morekeywords=[3]{SD, File},          % [3] -> bold orange
  morekeywords=[4]{open, exists, write, SoftwareSerial},      % [4] -> orange
}
%%%%%%%%%%%%%%%%%%%%%%%

\usepackage{siunitx} %pacchetto per le unita' di misura

%%%%%%%%%%%%%%%%%%%%%% per i flowchart
\usepackage{tikz}
\usetikzlibrary{arrows.meta}
\tikzset{%
  >={Latex[width=2mm,length=2mm]},
  % Specifications for style of nodes:
            base/.style = {rectangle, rounded corners, draw=black,
                           minimum width=4cm, minimum height=1cm,
                           text centered, font=\sffamily},
           round/.style = {circle, draw=black,
                           minimum width=4cm, minimum height=1cm,
                           text centered, font=\sffamily},
  			prog/.style = {base, fill=blue!30},
   			file/.style = {base, fill=red!30},
    activityRuns/.style = {base, fill=green!30},
         process/.style = {base, minimum width=2.5cm, fill=orange!15,
                           font=\ttfamily},
}
%%%%%%%%%%%%%%%%%%%%%%%%%%%%%%%%%%%%%%%%

\pagenumbering{arabic}
\pagestyle{plain}

% per non farlo anadre a capo ovunque
\usepackage[none]{hyphenat}
% per togliere gli ident all'inizio dei paragrafi
\setlength{\parindent}{0pt}
\begin{document}
 
\section{Prefazione}
introduciamo i programmi con uno schema generale della relazione logico temporale tra i programmi
\vspace{20pt}
\\
\begin{tikzpicture}[node distance=1.5cm, every node/.style={fill=white, font=\sffamily}, align=center]
% ora vanno specificate le posizioni 
%	\node (nome_blocco) 	[tipo_blocco, parentela] 	{testo_blocco};
	\node (prg_ard)	[
%
%\begin{tikzpicture}[node distance=1.5cm,
%    every node/.style={fill=white, font=\sffamily}, align=center]
%  % Specification of nodes (position, etc.)
%  \node (start)             [activityStarts]              {Activity starts};
%  \node (onCreateBlock)     [process, below of=start]          {onCreate()};
%  \node (onStartBlock)      [process, below of=onCreateBlock]   {onStart()};
%  \node (onResumeBlock)     [process, below of=onStartBlock]   {onResume()};
%  \node (activityRuns)      [activityRuns, below of=onResumeBlock]{Activity is running};
%  \node (onPauseBlock)      [process, below of=activityRuns, yshift=-1cm]{onPause()};
%  \node (onStopBlock)       [process, below of=onPauseBlock, yshift=-1cm]{onStop()};
%  \node (onDestroyBlock)    [process, below of=onStopBlock, yshift=-1cm]{onDestroy()};
%  \node (onRestartBlock)    [process, right of=onStartBlock, xshift=4cm]{onRestart()};
%  \node (ActivityEnds)      [startstop, left of=activityRuns, xshift=-4cm]{Process is killed};
%  \node (ActivityDestroyed) [startstop, below of=onDestroyBlock]{Activity is shut down};     
%  % Specification of lines between nodes specified above
%  % with aditional nodes for description 
%  \draw[->]             (start) -- (onCreateBlock);
%  \draw[->]     (onCreateBlock) -- (onStartBlock);
%  \draw[->]      (onStartBlock) -- (onResumeBlock);
%  \draw[->]     (onResumeBlock) -- (activityRuns);
%  \draw[->]      (activityRuns) -- node[text width=4cm]
%                                   {Another activity comes in
%                                    front of the activity} (onPauseBlock);
%  \draw[->,line width=5pt]      (onPauseBlock) -- node {The activity is no longer visible}
%                                   (onStopBlock);
%  \draw[->]       (onStopBlock) -- node {The activity is shut down by
%                                   user or system} (onDestroyBlock);
%  \draw[->]    (onRestartBlock) -- (onStartBlock);
%  \draw[->]       (onStopBlock) -| node[yshift=1.25cm, text width=3cm]
%                                   {The activity comes to the foreground}
%                                   (onRestartBlock);
%  \draw[->]    (onDestroyBlock) -- (ActivityDestroyed);
%  \draw[->]      (onPauseBlock) -| node(priorityXMemory)
%                                   {higher priority $\rightarrow$ more memory}
%                                   (ActivityEnds);
%  \draw           (onStopBlock) -| (priorityXMemory);
%  \draw[->]     (ActivityEnds)  |- node [yshift=-2cm, text width=3.1cm]
%                                    {User navigates back to the activity}
%                                    (onCreateBlock);
%  \draw[->] (onPauseBlock.east) -- ++(2.6,0) -- ++(0,2) -- ++(0,2) --                
%     node[xshift=1.2cm,yshift=-1.5cm, text width=2.5cm]
%     {The activity comes to the foreground}(onResumeBlock.east);
%\end{tikzpicture}
\vspace{30pt}
\\
cjusiaSISACHJENCL JWEKNFKC ENFWEJFWE KJJKj jee  e e e e

\end{document}