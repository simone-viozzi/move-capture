\documentclass[10pt,a4paper]{article}

%intestazione copiata

\usepackage[italian]{babel}
\usepackage{amsmath}
\usepackage{amsfonts}
\usepackage{amssymb}

\usepackage[left=1cm,right=1cm,top=1cm,bottom=2cm]{geometry}

\usepackage{txfonts}
\usepackage[T1]{fontenc}
\usepackage[utf8]{inputenc}

\usepackage{titlesec}
\setcounter{secnumdepth}{4}
\titleformat{\paragraph}{\normalfont\normalsize\bfseries}{\theparagraph}{1em}{}
\titlespacing*{\paragraph}{0pt}{3.25ex plus 1ex minus .2ex}{1.5ex plus .2ex}

%per le immagini
\usepackage{graphicx}
\usepackage{subcaption}
\usepackage{wrapfig}

%per i link
\usepackage{hyperref} 

%%%%%%%% per il codice c++
\usepackage{textcomp}
\usepackage{listings}          % for creating language style
\usepackage{listingsutf8}
\input{arduinoLanguage.tex}    % adds the arduino language listing
\definecolor{commentgreen}{RGB}{2,112,10}
\definecolor{eminence}{RGB}{108,48,130}
\definecolor{weborange}{RGB}{255,165,0}
\definecolor{frenchplum}{RGB}{129,20,83}


%% Define an Arduino style fore use later %%
\lstdefinestyle{myArduino}{
  language=Arduino,
    %% Add other words needing highlighting below %%
    morekeywords=[1]{},                  % [1] -> dark green
    morekeywords=[2]{FILE_WRITE},        % [2] -> light blue
    morekeywords=[3]{SD, File},          % [3] -> bold orange
    morekeywords=[4]{open, exists, write, SoftwareSerial},      % [4] -> orange
    frame=tb,    
    inputencoding=utf8,
    extendedchars=true,
    literate={è}{{\`{e}}}{1},
    breaklines=true,  
}

\lstdefinestyle{mycpp}{
    language=C++,
    inputencoding=utf8,
    extendedchars=true,
    literate={è}{{\`{e}}}{1},
    %escapeinside={(*******}{*******)}
    %escapechar=\£,
    %escapeinside=~~,
    frame=tb,
    tabsize=2,
    mathescape=false,
    breaklines=true,                    % wordwrapping
    postbreak=\mbox{\textcolor{red}{$\hookrightarrow$}\space},         
    basicstyle=\fontsize{9}{11}\ttfamily,
    backgroundcolor=\color{light-gray},
    xleftmargin=.25in,
    showstringspaces=false,
    numbers=left,                    
    numbersep=5pt,                   
    %numberstyle=\color{arduinoGrey},    
    %stepnumber=2, 
    %upquote=true,
    commentstyle=\color{commentgreen},
    keywordstyle=\color{eminence},
    stringstyle=\color{red},
    basicstyle=\small\ttfamily, % basic font setting
    emph={int,char,double,float,unsigned,void,bool},
    emphstyle={\color{blue}},
    % keyword highlighting
    classoffset=1, % starting new class
    otherkeywords={>,<,.,;,-,!,=,~},
    morekeywords={>,<,.,;,-,!,=,~},
    keywordstyle=\color{weborange},
    classoffset=0,
}


\lstdefinestyle{myoutput}
{
    inputencoding=utf8,
    extendedchars=true,
    literate={è}{{\`{e}}}{1},
    tabsize=2,
    frame=tb,
    breaklines=true,                    % wordwrapping
    postbreak=\mbox{\textcolor{red}{$\hookrightarrow$}\space},         
    basicstyle=\fontsize{9}{11}\ttfamily,
    backgroundcolor=\color{light-gray},
    xleftmargin=.25in,
    showstringspaces=false,
    numbers=left,                    
    numbersep=5pt, 
}
%%%%%%%%%%%%%%%%%%%%%%%


\usepackage{siunitx} %pacchetto per le unita' di misura

%%%%%%%%%%%%%%%%%%%%%% per i flowchart
\usepackage{xcolor}
\usepackage{tikz}
\usetikzlibrary{shapes,arrows}
\usetikzlibrary{arrows.meta}
\usetikzlibrary{positioning}
\usetikzlibrary{shapes.geometric}
\usepgflibrary{shapes.symbols}
\usetikzlibrary{shapes.multipart}

\tikzset{%
  >={Latex[width=2mm,length=2mm]},
  % Specifications for style of nodes:
            rect/.style = {rectangle, rounded corners, draw=black,
                           minimum width=4cm, minimum height=1cm,
                           text centered, font=\sffamily},
           round/.style = {ellipse, draw, draw=black,
                           minimum width=4cm, minimum height=1cm,
                           text centered, font=\sffamily},
       smallrect/.style = {rectangle, rounded corners, draw=black,
                           minimum width=2cm, minimum height=1cm,
                           text centered, font=\sffamily},
 smallrectsplit4/.style = {rectangle split, rectangle split parts=4, 
	                       rectangle split part fill={green!30, none, none, none},
	                       align=center,
	                       rounded corners, draw=black,
                           minimum width=2cm, minimum height=1cm,
                           text centered, font=\sffamily},
}

%\tikzset{%
%    >={Latex[width=2mm,length=2mm]},
%      % Specifications for style of nodes:
%         declare/.style = {trapezium,draw=black, minimum width=4cm, minimum height=1cm, 
%                                trapezium right angle=-70, trapezium left angle=70,
%                                minimum width=4cm, minimum height=1cm,
%                                text centered, font=\sffamily},
%           start/.style = {ellipse, draw, draw=black, minimum width=4cm, 
%                                minimum height=1cm, text centered, font=\sffamily},
%            cond/.style = {diamond, aspect=2, draw, draw=black,
%                                minimum width=4cm, minimum height=1cm,
%                                text centered, font=\sffamily},
%            rect/.style = {rectangle, draw, draw=black,
%                                minimum width=4cm, minimum height=1cm,
%                                text centered, font=\sffamily},
%}
%%%%%%%%%%%%%%%%%%%%%%%%%%%%%%%%%%%%%%%
% 

\pagenumbering{arabic}
\pagestyle{plain}

% per non farlo anadre a capo ovunque 
% va in conflitto con quello che lo fa andare a capo nel codice quindi attenzione <--------
%\usepackage[none]{hyphenat}


% per togliere gli ident all'inizio dei paragrafi
\setlength{\parindent}{0pt}






\begin{document}

\subsection{Programma che si occupa di post-elaborare i dati}
Dopo aver acquisito i dati di arduino e del kinect bisogna effettuare una post elaborazione per ottenere coppie di valori riferiti a circa lo stesso istante temporale. 
Il programma usato per fare ci\`o riprende molte cose dal codice del programma "prg\_get\_data", le librerie usate sono le stesse ed anche le classi "kin\_file\_manager", "kin\_file\_manager" e "data\_structure.h".
L'unica classe nuova \`e "make\_dataset":
\begin{lstlisting}[style=mycpp, caption=classe make\_dataset, captionpos=b]
class make_dataset
{
private:
  // dichiaro i puntatori alle classi per la gestione dei file cos\`i da renderle disponibili per tutta la classe
  ard_file_manager *f_a;
  kin_file_manager *f_k;

  // delta di accettazione in millisecondi
  const int range = 35; 

  // dichiaro 2 liste per conservare i dati 
  list<arduino_data> a_dat;
  list<kinect_data> k_dat;

  // devo dichiararare il numero di ingressi e di uscite della rete
  static const std::size_t n_in = 9;
  static const std::size_t n_out = 4;

  // dichiaro una lista di tipo dataset_data che verr\`a riempita con i dati accoppiati
  list<dataset_data<float, n_in, float, n_out>> dataset;

  // questa funzione si occupa di scrivere il dataset sul file
  void write_dataset(string dataset_file_name){...}

  // restituisce 1 solo se i 2 tempi che gli vengono passati differiscono meno della soglia impostata sopra
  bool check_sync(uint64_t t1, uint64_t t2){...}

  // questo metodo sposta i dati dalle strutture "arduino_data" e "kinect_data" alla struttura "dataset" dichiarata globalmente nella classe
  void transfer_data_to_dataset(arduino_data* ard_iter, kinect_data* kin_iter){...}

  // restituisce la differenza temporale tra i tempi passati alla funzione
  uint32_t time_distance(uint32_t t1, uint32_t t2){...}

  // sincronizza i dati caricati in "a_dat", "k_dat" e li sposta in "dataset" attraverso la funzione "transfer_data_to_dataset". Questa funzione non tiene conto dei "cloni", nella sezione successiva vi \'e una spiegazione pi\'u dettagliata. 
  void sync_with_clones(){...}

  // stessa cosa della funzione precedente ma eliminando i cloni.
  void sync_without_clones(){...}

  // carica i dati dai file usando "f_a" e "f_k", sposta i dati acquisiti in "a_dat" e k_dat"
  void load_data(){...}

public:
  // l'unica funzione accessibile \`e il costruttore che provvede ad inizializzare i  file manager, caricare i dati, sincronizzarli (con o senza cloni) e scrivere il file del dataset con i dati appena generati.
  make_dataset(string ard_file_name, string kin_file_name, string dataset_file_name, bool want_clones)
  {
    f_a = new ard_file_manager(ard_file_name, std::ios::in);
    f_k = new kin_file_manager(kin_file_name, std::ios::in);

    load_data();
    
    if (want_clones)
    {
      sync_with_clones();
    }
    else
    {
      sync_without_clones();
    }

    write_dataset(dataset_file_name);

  }

  // distruttore della classe
  ~make_dataset(){...}

};
\end{lstlisting}
In questa classe le 2 funzioni principali sono "sync\_with\_clones" e "sync\_without\_clones" che provvedono a costruire le coppie di dati ed a salvarle nella struttura dati "dataset".
Analizziamo la funzione "sync\_with\_clones":
\begin{lstlisting}[style=mycpp, caption=funzione sync\_with\_clones, captionpos=b, label={lst:syncclones}]
void sync_with_clones()
{
  uint64_t time_old_good_pair = 0;

  uint64_t time_old_ard = 0;
  uint64_t time_old_kin = 0;

  auto ard_iter = a_dat.begin();
  auto kin_iter = k_dat.begin();

  uint32_t ard_cont = 0;
  uint32_t kin_cont = 0;

  uint32_t old_ard_cont = 0;
  uint32_t old_kin_cont = 0;

  while (true)
  {
    // controllo la sincronia 
    if (check_sync(ard_iter->frame_time, kin_iter->frame_time))
    {
      // aggiungo il dato al dataset
      transfer_data_to_dataset(&(*ard_iter), &(*kin_iter));

      // stampo delle info utili a capire la qualit\`a dei dati acquisiti
      cout << abs((long long)(ard_iter->frame_time - kin_iter->frame_time)) << "  kin: " << kin_cont << " ard: " << ard_cont << endl;
      cout << "time since another good pair: " << ((ard_iter->frame_time + kin_iter->frame_time) / 2 - time_old_good_pair) << endl;
      cout << "ard frame rate: " << (1.0f / ((ard_iter->frame_time - time_old_ard) / 1000.0f)) << "Hz" << endl;
      cout << "kin frame rate: " << (1.0f / ((kin_iter->frame_time - time_old_kin) / 1000.0f)) << "Hz" << endl;

      // salvo i tempi
      time_old_good_pair = (ard_iter->frame_time + kin_iter->frame_time) / 2;
      time_old_ard = ard_iter->frame_time;
      time_old_kin = kin_iter->frame_time;

      // controllo se ci sono pi\`u dati che si accoppiano con uno solo (solo debug)
      if ((kin_cont == old_kin_cont) || (ard_cont == old_ard_cont))
      {
        cout << "dati multipli nello stesso range" << endl;
      }
      else
      {
        old_kin_cont = kin_cont;
        old_ard_cont = ard_cont;
      }

      cout << endl;

    }
    else
    {
      cout << "dato scartato" << endl << endl;
    }

    // incremento chi tra i 2 dati \`e il pi\`u vecchio
    if (ard_iter->frame_time >= kin_iter->frame_time)
    {
      kin_iter++;
      kin_cont++;
    }
    else
    {
      ard_iter++;
      ard_cont++;
    }

    // se sono arrivato alla fine esco dal ciclo
    if ((kin_iter == k_dat.end()) || (ard_iter == a_dat.end())) { break; }
  }
}
\end{lstlisting}
Questa prima funzione effettua il pair dei dati senza per\`o tener conto che, dati i differenti frame rate di arduino e del kinect, un singolo dato del kinect potrebbe accoppiarsi con pi\`u di un dato di arduino e viceversa. Questi "cloni" interferiscono con il processo di apprendimento della rete e quindi si \`e passati alla funzione "sync\_without\_clones" che effettua la stessa operazione ma tenendo conto che potrebbero esserci pi\`u dati di un tipo accoppiati con uno dell'altro e salva solo la coppia che ha la differenza di tempo minore.
\begin{lstlisting}[style=mycpp, caption=funzione sync\_without\_clones, captionpos=b]
void sync_without_clones()
{
  uint64_t time_old_good_pair = 0;

  uint64_t time_old_ard = 0;
  uint64_t time_old_kin = 0;

  auto ard_iter = a_dat.begin();
  auto kin_iter = k_dat.begin();

  uint32_t ard_cont = 0;
  uint32_t kin_cont = 0;

  uint32_t old_ard_cont = 0;
  uint32_t old_kin_cont = 0;

  kinect_data k_tmp = *kin_iter;
  arduino_data a_tmp = *ard_iter;

  for(;;)
  {
    // controllo la sincronia
    if (check_sync(ard_iter->frame_time, kin_iter->frame_time))
    {
      // stampo i tempi per valutere la qualit\`a del dataset
      cout << "common " << abs((long long)(ard_iter->frame_time - kin_iter->frame_time)) << "  kin: " << kin_iter->contatore << " ard: " << ard_iter->contatore << endl;
      cout << "ard frame rate: " << (1.0f / ((ard_iter->frame_time - time_old_ard) / 1000.0f)) << "Hz" << endl;
      cout << "kin frame rate: " << (1.0f / ((kin_iter->frame_time - time_old_kin) / 1000.0f)) << "Hz" << endl;

      // salvo i tempi attuali
      time_old_ard = ard_iter->frame_time;
      time_old_kin = kin_iter->frame_time;

      // se il contatore del kinect non \`e variato dall'ultima volta significa che ci sono pi\`u dati riferiti ad arduino che si accoppiano ad un solo dato del kinect
      if (kin_cont == old_kin_cont)
      {
      
        cout << "multi ard in the same kin" << endl;
        if (time_distance(kin_iter->frame_time, ard_iter->frame_time) < time_distance(k_tmp.frame_time, a_tmp.frame_time))
        {
          a_tmp = *ard_iter;
        }
      }
      // se il contatore dei arduino non \`e variato dall'ultima volta significa che ci sono pi\`u dati riferiti al kinect che si accoppiano ad un solo dato di arduino
      else if (ard_cont == old_ard_cont)
      {
        cout << "multi kin in the same ard" << endl;
        if (time_distance(kin_iter->frame_time, ard_iter->frame_time) < time_distance(k_tmp.frame_time, a_tmp.frame_time))
        {
          k_tmp = *kin_iter;
        }
      }
      \\ se entrambi sono variati significa che la sequenza \`e finita e posso salvare i dati
      else
      {
        cout << "sequence reset" << endl;

        // salvo la coppia salvata
        transfer_data_to_dataset(&a_tmp, &k_tmp);

        // stampo i tempi per le valutazioni
        cout << "best " << abs((long long)(a_tmp.frame_time - k_tmp.frame_time)) << "  kin: " << k_tmp.contatore << " ard: " << a_tmp.contatore << endl;
        cout << "time since another good pair: " << ((a_tmp.frame_time + k_tmp.frame_time) / 2 - time_old_good_pair) << endl;
        time_old_good_pair = (a_tmp.frame_time + k_tmp.frame_time) / 2;

        // salvo i dati correnti e i contatori correnti
        old_kin_cont = kin_cont;
        old_ard_cont = ard_cont;
        a_tmp = *ard_iter;
        k_tmp = *kin_iter;
      }

      cout << endl;

    }
    else
    {
      cout << "dato scartato" << endl << endl;
    }

    // incremento chi tra i 2 dati \`e il pi\`u vecchio
    if (ard_iter->frame_time >= kin_iter->frame_time)
    {
      kin_iter++;
      kin_cont++;
    }
    else
    {
      ard_iter++;
      ard_cont++;
    }

    // se sono arrivato alla fine dei dati esco
    if ((kin_iter == k_dat.end()) || (ard_iter == a_dat.end())) { break; }
  }
}
\end{lstlisting}


La funzione che si occupa di acquisire i nomi dei file e altro dall'utente \`e "post\_elaborate":
\begin{lstlisting}[style=mycpp, caption=funzione "post\_elaborate", captionpos=b]
void post_elaborate()
{
  string ard_file_mane = "";
  string kin_file_mane = "";
  string dataset_file_name = "";
  string tmp = "";
  bool clones = false;
  
  // acquisisco i nomi dei file dei dati di arduino e del kinect
  cout << "insert the input file names: " << endl
    << "\t arduino -> ";
  std::getline(std::cin, ard_file_mane);
  cout << "\t kinekt -> ";
  std::getline(std::cin, kin_file_mane);

  // acquisisco il nome del file in cui verr\`a scritto il dataset
  cout << "insert the ouptut file name -> : ";
  std::getline(std::cin, dataset_file_name);

  // chiedo se si vuole il dataset con o senza cloni
  cout << "do you want clones in the data? [y/n] -> : ";
  std::getline(std::cin, tmp);

  if (tmp == "y")
  {
    clones = true;
  }
  
  // chiamo la classe make_dataset passandogli i dati appena acquisiti
  make_dataset(ard_file_mane, kin_file_mane, dataset_file_name, clones);
}
\end{lstlisting}



\end{document}
























