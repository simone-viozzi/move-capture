\documentclass[10pt,a4paper]{article}

%intestazione copiata


\usepackage[italian]{babel}
\usepackage{amsmath}
\usepackage{amsfonts}
\usepackage{amssymb}

\usepackage[left=1cm,right=1cm,top=1cm,bottom=2cm]{geometry}

\usepackage{txfonts}
\usepackage[T1]{fontenc}
\usepackage[utf8]{inputenc}

\usepackage{titlesec}
\setcounter{secnumdepth}{4}
\titleformat{\paragraph}{\normalfont\normalsize\bfseries}{\theparagraph}{1em}{}
\titlespacing*{\paragraph}{0pt}{3.25ex plus 1ex minus .2ex}{1.5ex plus .2ex}

\usepackage{graphicx}
\usepackage{subcaption}

\usepackage{wrapfig}

%%%%%%%% per il codice c++
\usepackage{listings}          % for creating language style
\input{arduinoLanguage.tex}    % adds the arduino language listing

%% Define an Arduino style fore use later %%
\lstdefinestyle{myArduino}{
  language=Arduino,
  %% Add other words needing highlighting below %%
  morekeywords=[1]{},                  % [1] -> dark green
  morekeywords=[2]{FILE_WRITE},        % [2] -> light blue
  morekeywords=[3]{SD, File},          % [3] -> bold orange
  morekeywords=[4]{open, exists, write, SoftwareSerial},      % [4] -> orange
}
%%%%%%%%%%%%%%%%%%%%%%%

\usepackage{siunitx} %pacchetto per le unita' di misura

%%%%%%%%%%%%%%%%%%%%%% per i flowchart
\usepackage{xcolor}
\usepackage{tikz}
\usetikzlibrary{shapes,arrows}
\usetikzlibrary{arrows.meta}
\tikzset{%
  >={Latex[width=2mm,length=2mm]},
  % Specifications for style of nodes:
            rect/.style = {rectangle, rounded corners, draw=black,
                           minimum width=4cm, minimum height=1cm,
                           text centered, font=\sffamily},
           round/.style = {ellipse, draw, draw=black,
                           minimum width=4cm, minimum height=1cm,
                           text centered, font=\sffamily},
}
%%%%%%%%%%%%%%%%%%%%%%%%%%%%%%%%%%%%%%%%

\pagenumbering{arabic}
\pagestyle{plain}

% per non farlo anadre a capo ovunque
\usepackage[none]{hyphenat}
% per togliere gli ident all'inizio dei paragrafi
\setlength{\parindent}{0pt}





\begin{document}
 
\subsection{Prefazione}
Introduciamo i programmi con uno schema generale della relazione logico temporale tra i programmi:
per prima cosa abbiamo dovuto acquisire un dataset ed usarlo per addesrare la rete neurale
\begin{figure}[h]
\centering
\begin{tikzpicture}[node distance=2cm, align=center, every node/.style={fill=white, font=\sffamily}, align=center]
%	\draw[help lines] (-10,-10) grid (1,1);
% ora vanno specificate le posizioni 
%	\node (nome_blocco) 	[tipo_blocco, parentela] 		
%								{testo_blocco};
	\node (prg_ard)			[rect, fill=green!30] 							
								{programma arduino};
	\node (prg_get_data)	[rect, right of=prg_ard, xshift=+4cm, fill=yellow!30] 
								{programma di acquisizione dati};
	\node (file_data_kin)	[round, below of=prg_get_data, xshift=+2.2cm, yshift=-0.75cm, fill=gray!30]	
								{file dati kinect};
	\node (file_data_ard)	[round, below of=prg_get_data, xshift=-2.2cm, yshift=-0.75cm, fill=gray!30]	
								{file dati arduino};
	\node (prg_elab_data)	[rect, below of=prg_get_data, yshift=-3.5cm, fill=yellow!30]
								{programma di post\\ elaborazione dei dati};
	\node (file_dataset)	[round, below of=prg_elab_data, fill=gray!30]	
								{dataset};
	\node (prg_addesr)		[rect, below of=file_dataset, fill=yellow!30]
								{programma di addestramento\\ della rete (OpenLearn)};
	\node (file_pesi_rete)	[round, below of=prg_addesr, fill=gray!30]	
								{modello addestrato};
% per le freccie si fa
%	\daraw[tipo_freccia] (origine) "tipo connessione" "eventuale scritta" (destinazione)			
	\draw[blue, line width=1mm, arrows={[scale=1.75,blue]<->[scale=1.75,blue]}]	
				(prg_ard)		-- 					(prg_get_data);
	\draw[->, red]	(prg_get_data)	to [in=90,out=-90]	(file_data_kin);
	\draw[->, red]	(prg_get_data)	to [in=90,out=-90]	(file_data_ard);
	\draw[->, cyan]	(file_data_kin)	to [in=90,out=-90]	(prg_elab_data);
	\draw[->, cyan]	(file_data_ard)	to [in=90,out=-90] 	(prg_elab_data);
	\draw[->, red]	(prg_elab_data)	-- 					(file_dataset);
	\draw[->, cyan]	(file_dataset)	-- 					(prg_addesr);
	\draw[->, red]	(prg_addesr)	-- 					(file_pesi_rete);
%
\matrix [draw,above] at (current bounding box.south west) {
  \node [shape=circle, fill=green, label=right:programma per la scheda arduino] {}; \\
  \node [shape=circle, fill=yellow, label=right:programma per pc] {}; \\
  \node [shape=circle, fill=gray, label=right:file di testo] {}; \\
  \draw[->, cyan, text=black] ++(-1em, -0em) -- ++(3em, -0em) node[right] {file in input}; \\
  \draw[->, red, text=black] ++(-1em, -0em) -- ++(3em, -0em) node[right] {file in output}; \\
  \draw[blue, text=black, line width=1mm, arrows={[scale=1.75,blue]<->[scale=1.75,blue]}]
  		 ++(-1em, -0em) -- ++(3em, 0) node[right] {in esecuzione contemporanea}; \\
};
\end{tikzpicture}
\caption{i programmi usati per ottenere i pesi della rete addestrata (?)}
\end{figure}
\\
una volta addestrata la rete si possono usare i pesi cos\`i ottenuti per visualizzare in tempo reale il pupo che si muove(???)
\begin{figure}[h]
\centering
\begin{tikzpicture}[node distance=2cm, align=center, every node/.style={fill=white, font=\sffamily}, align=center]
%	\draw[help lines] (-10,-10) grid (1,1);
% ora vanno specificate le posizioni 
%	\node (nome_blocco) 	[tipo_blocco, parentela] 		
%								{testo_blocco};
	\node (prg_ard)			[rect, fill=green!30] 							
								{programma arduino};
	\node (prg_main)	[rect, right of=prg_ard, fill=yellow!30, xshift=4cm] 
								{main (??)};
	\node (Modello)	[round, above of=prg_main, fill=gray!30]	
								{Modello addestrato};	
	\node (prg_unity)	[rect, right of=prg_main, fill=yellow!30, xshift=4cm]
								{Interfaccia grafica (Unity3D)};
% per le freccie si fa
%	\daraw[tipo_freccia] (origine) "tipo connessione" "eventuale scritta" (destinazione)			
	\draw[blue, line width=1mm, arrows={[scale=1.75,blue]<->[scale=1.75,blue]}]	
					(prg_ard)		-- 	(prg_main);
	\draw[blue, line width=1mm, arrows={[scale=1.75,blue]<->[scale=1.75,blue]}]	
					(prg_main)		-- 	(prg_unity);
	\draw[->, cyan]	(Modello)		--	(prg_main);
\end{tikzpicture}
\caption{una volta addestrata la rete si pu\`o vedere il pupo muoversi in tempo reale}
\end{figure}
\\
TODO aggiungere ci\`o che manca


\end{document}