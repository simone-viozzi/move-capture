\documentclass[10pt,a4paper]{article}

\usepackage[italian]{babel}
\usepackage{amsmath}
\usepackage{amsfonts}
\usepackage{amssymb}

\usepackage[left=1cm,right=1cm,top=1cm,bottom=2cm]{geometry}

\usepackage{txfonts}
\usepackage[T1]{fontenc}
\usepackage[utf8]{inputenc}

\usepackage{titlesec}
\setcounter{secnumdepth}{4}
\titleformat{\paragraph}{\normalfont\normalsize\bfseries}{\theparagraph}{1em}{}
\titlespacing*{\paragraph}{0pt}{3.25ex plus 1ex minus .2ex}{1.5ex plus .2ex}

\usepackage{graphicx}
\usepackage{subcaption}

\usepackage{wrapfig}

%%%%%%%% per il codice c++
\usepackage{textcomp}
\usepackage{listings}          % for creating language style
\usepackage{listingsutf8}
 %%%%%%%%%%%%%%%%%%%%%%%%%%%%%%%%%%%%%%%%%%%%%%%%%%%%%%%%%%%%%%%%%%%%%%%%%%%%%%%% 
%%% ~ Arduino Language - Arduino IDE Colors ~                                  %%%
%%%                                                                            %%%
%%% Kyle Rocha-Brownell | 10/2/2017 | No Licence                               %%%
%%% -------------------------------------------------------------------------- %%%
%%%                                                                            %%%
%%% Place this file in your working directory (next to the latex file you're   %%%
%%% working on).  To add it to your project, place:                            %%%
%%%     %%%%%%%%%%%%%%%%%%%%%%%%%%%%%%%%%%%%%%%%%%%%%%%%%%%%%%%%%%%%%%%%%%%%%%%%%%%%%%%% 
%%% ~ Arduino Language - Arduino IDE Colors ~                                  %%%
%%%                                                                            %%%
%%% Kyle Rocha-Brownell | 10/2/2017 | No Licence                               %%%
%%% -------------------------------------------------------------------------- %%%
%%%                                                                            %%%
%%% Place this file in your working directory (next to the latex file you're   %%%
%%% working on).  To add it to your project, place:                            %%%
%%%     %%%%%%%%%%%%%%%%%%%%%%%%%%%%%%%%%%%%%%%%%%%%%%%%%%%%%%%%%%%%%%%%%%%%%%%%%%%%%%%% 
%%% ~ Arduino Language - Arduino IDE Colors ~                                  %%%
%%%                                                                            %%%
%%% Kyle Rocha-Brownell | 10/2/2017 | No Licence                               %%%
%%% -------------------------------------------------------------------------- %%%
%%%                                                                            %%%
%%% Place this file in your working directory (next to the latex file you're   %%%
%%% working on).  To add it to your project, place:                            %%%
%%%    \input{arduinoLanguage.tex}                                             %%%
%%% somewhere before \begin{document} in your latex file.                      %%%
%%%                                                                            %%%
%%% In your document, place your arduino code between:                         %%%
%%%   \begin{lstlisting}[language=Arduino]                                     %%%
%%% and:                                                                       %%%
%%%   \end{lstlisting}                                                         %%%
%%%                                                                            %%%
%%% Or create your own style to add non-built-in functions and variables.      %%%
%%%                                                                            %%%
 %%%%%%%%%%%%%%%%%%%%%%%%%%%%%%%%%%%%%%%%%%%%%%%%%%%%%%%%%%%%%%%%%%%%%%%%%%%%%%%% 

\usepackage{color}
\usepackage{listings}    
\usepackage{courier}
\usepackage{listingsutf8}

%%% Define Custom IDE Colors %%%
\definecolor{arduinoGreen}    {rgb} {0.17, 0.43, 0.01}
\definecolor{arduinoGrey}     {rgb} {0.47, 0.47, 0.33}
\definecolor{arduinoOrange}   {rgb} {0.8 , 0.4 , 0   }
\definecolor{arduinoBlue}     {rgb} {0.01, 0.61, 0.98}
\definecolor{arduinoDarkBlue} {rgb} {0.0 , 0.2 , 0.5 }
\definecolor{light-gray}{gray}{0.95}

%%% Define Arduino Language %%%
\lstdefinelanguage{Arduino}{
  language=C++, % begin with default C++ settings 
%
%
  %%% Keyword Color Group 1 %%%  (called KEYWORD3 by arduino)
  keywordstyle=\color{arduinoGreen},   
  deletekeywords={  % remove all arduino keywords that might be in c++
                break, case, override, final, continue, default, do, else, for, 
                if, return, goto, switch, throw, try, while, setup, loop, export, 
                not, or, and, xor, include, define, elif, else, error, if, ifdef, 
                ifndef, pragma, warning,
                HIGH, LOW, INPUT, INPUT_PULLUP, OUTPUT, DEC, BIN, HEX, OCT, PI, 
                HALF_PI, TWO_PI, LSBFIRST, MSBFIRST, CHANGE, FALLING, RISING, 
                DEFAULT, EXTERNAL, INTERNAL, INTERNAL1V1, INTERNAL2V56, LED_BUILTIN, 
                LED_BUILTIN_RX, LED_BUILTIN_TX, DIGITAL_MESSAGE, FIRMATA_STRING, 
                ANALOG_MESSAGE, REPORT_DIGITAL, REPORT_ANALOG, SET_PIN_MODE, 
                SYSTEM_RESET, SYSEX_START, auto, int8_t, int16_t, int32_t, int64_t, 
                uint8_t, uint16_t, uint32_t, uint64_t, char16_t, char32_t, operator, 
                enum, delete, bool, boolean, byte, char, const, false, float, double, 
                null, NULL, int, long, new, private, protected, public, short, 
                signed, static, volatile, String, void, true, unsigned, word, array, 
                sizeof, dynamic_cast, typedef, const_cast, struct, static_cast, union, 
                friend, extern, class, reinterpret_cast, register, explicit, inline, 
                _Bool, complex, _Complex, _Imaginary, atomic_bool, atomic_char, 
                atomic_schar, atomic_uchar, atomic_short, atomic_ushort, atomic_int, 
                atomic_uint, atomic_long, atomic_ulong, atomic_llong, atomic_ullong, 
                virtual, PROGMEM,
                Serial, Serial1, Serial2, Serial3, SerialUSB, Keyboard, Mouse,
                abs, acos, asin, atan, atan2, ceil, constrain, cos, degrees, exp, 
                floor, log, map, max, min, radians, random, randomSeed, round, sin, 
                sq, sqrt, tan, pow, bitRead, bitWrite, bitSet, bitClear, bit, 
                highByte, lowByte, analogReference, analogRead, 
                analogReadResolution, analogWrite, analogWriteResolution, 
                attachInterrupt, detachInterrupt, digitalPinToInterrupt, delay, 
                delayMicroseconds, digitalWrite, digitalRead, interrupts, millis, 
                micros, noInterrupts, noTone, pinMode, pulseIn, pulseInLong, shiftIn, 
                shiftOut, tone, yield, Stream, begin, end, peek, read, print, 
                println, available, availableForWrite, flush, setTimeout, find, 
                findUntil, parseInt, parseFloat, readBytes, readBytesUntil, readString, 
                readStringUntil, trim, toUpperCase, toLowerCase, charAt, compareTo, 
                concat, endsWith, startsWith, equals, equalsIgnoreCase, getBytes, 
                indexOf, lastIndexOf, length, replace, setCharAt, substring, 
                toCharArray, toInt, press, release, releaseAll, accept, click, move, 
                isPressed, isAlphaNumeric, isAlpha, isAscii, isWhitespace, isControl, 
                isDigit, isGraph, isLowerCase, isPrintable, isPunct, isSpace, 
                isUpperCase, isHexadecimalDigit, 
                }, 
  morekeywords={   % add arduino structures to group 1
                break, case, override, final, continue, default, do, else, for, 
                if, return, goto, switch, throw, try, while, setup, loop, export, 
                not, or, and, xor, include, define, elif, else, error, if, ifdef, 
                ifndef, pragma, warning,
                }, 
% 
%
  %%% Keyword Color Group 2 %%%  (called LITERAL1 by arduino)
  keywordstyle=[2]\color{arduinoBlue},   
  keywords=[2]{   % add variables and dataTypes as 2nd group  
                HIGH, LOW, INPUT, INPUT_PULLUP, OUTPUT, DEC, BIN, HEX, OCT, PI, 
                HALF_PI, TWO_PI, LSBFIRST, MSBFIRST, CHANGE, FALLING, RISING, 
                DEFAULT, EXTERNAL, INTERNAL, INTERNAL1V1, INTERNAL2V56, LED_BUILTIN, 
                LED_BUILTIN_RX, LED_BUILTIN_TX, DIGITAL_MESSAGE, FIRMATA_STRING, 
                ANALOG_MESSAGE, REPORT_DIGITAL, REPORT_ANALOG, SET_PIN_MODE, 
                SYSTEM_RESET, SYSEX_START, auto, int8_t, int16_t, int32_t, int64_t, 
                uint8_t, uint16_t, uint32_t, uint64_t, char16_t, char32_t, operator, 
                enum, delete, bool, boolean, byte, char, const, false, float, double, 
                null, NULL, int, long, new, private, protected, public, short, 
                signed, static, volatile, String, void, true, unsigned, word, array, 
                sizeof, dynamic_cast, typedef, const_cast, struct, static_cast, union, 
                friend, extern, class, reinterpret_cast, register, explicit, inline, 
                _Bool, complex, _Complex, _Imaginary, atomic_bool, atomic_char, 
                atomic_schar, atomic_uchar, atomic_short, atomic_ushort, atomic_int, 
                atomic_uint, atomic_long, atomic_ulong, atomic_llong, atomic_ullong, 
                virtual, PROGMEM,
                },  
% 
%
  %%% Keyword Color Group 3 %%%  (called KEYWORD1 by arduino)
  keywordstyle=[3]\bfseries\color{arduinoOrange},
  keywords=[3]{  % add built-in functions as a 3rd group
                Serial, Serial1, Serial2, Serial3, SerialUSB, Keyboard, Mouse,
                },      
%
%
  %%% Keyword Color Group 4 %%%  (called KEYWORD2 by arduino)
  keywordstyle=[4]\color{arduinoOrange},
  keywords=[4]{  % add more built-in functions as a 4th group
                abs, acos, asin, atan, atan2, ceil, constrain, cos, degrees, exp, 
                floor, log, map, max, min, radians, random, randomSeed, round, sin, 
                sq, sqrt, tan, pow, bitRead, bitWrite, bitSet, bitClear, bit, 
                highByte, lowByte, analogReference, analogRead, 
                analogReadResolution, analogWrite, analogWriteResolution, 
                attachInterrupt, detachInterrupt, digitalPinToInterrupt, delay, 
                delayMicroseconds, digitalWrite, digitalRead, interrupts, millis, 
                micros, noInterrupts, noTone, pinMode, pulseIn, pulseInLong, shiftIn, 
                shiftOut, tone, yield, Stream, begin, end, peek, read, print, 
                println, available, availableForWrite, flush, setTimeout, find, 
                findUntil, parseInt, parseFloat, readBytes, readBytesUntil, readString, 
                readStringUntil, trim, toUpperCase, toLowerCase, charAt, compareTo, 
                concat, endsWith, startsWith, equals, equalsIgnoreCase, getBytes, 
                indexOf, lastIndexOf, length, replace, setCharAt, substring, 
                toCharArray, toInt, press, release, releaseAll, accept, click, move, 
                isPressed, isAlphaNumeric, isAlpha, isAscii, isWhitespace, isControl, 
                isDigit, isGraph, isLowerCase, isPrintable, isPunct, isSpace, 
                isUpperCase, isHexadecimalDigit, 
                },      
%
  extendedchars=true,
%
  %%% Set Other Colors %%%
  stringstyle=\color{arduinoDarkBlue},
  showstringspaces=false,  
  commentstyle=\color{arduinoGrey},    
%          
%   
  %%%% Line Numbering %%%%
  numbers=left,                    
  numbersep=5pt,                   
  numberstyle=\color{arduinoGrey},    
  %stepnumber=2,                      % show every 2 line numbers
%
%
  %%%% Code Box Style %%%%
  breaklines=true,                    % wordwrapping
  tabsize=2,         
  basicstyle=\fontsize{9}{11}\ttfamily,
  backgroundcolor=\color{light-gray},
  xleftmargin=.35in
}                                             %%%
%%% somewhere before \begin{document} in your latex file.                      %%%
%%%                                                                            %%%
%%% In your document, place your arduino code between:                         %%%
%%%   \begin{lstlisting}[language=Arduino]                                     %%%
%%% and:                                                                       %%%
%%%   \end{lstlisting}                                                         %%%
%%%                                                                            %%%
%%% Or create your own style to add non-built-in functions and variables.      %%%
%%%                                                                            %%%
 %%%%%%%%%%%%%%%%%%%%%%%%%%%%%%%%%%%%%%%%%%%%%%%%%%%%%%%%%%%%%%%%%%%%%%%%%%%%%%%% 

\usepackage{color}
\usepackage{listings}    
\usepackage{courier}
\usepackage{listingsutf8}

%%% Define Custom IDE Colors %%%
\definecolor{arduinoGreen}    {rgb} {0.17, 0.43, 0.01}
\definecolor{arduinoGrey}     {rgb} {0.47, 0.47, 0.33}
\definecolor{arduinoOrange}   {rgb} {0.8 , 0.4 , 0   }
\definecolor{arduinoBlue}     {rgb} {0.01, 0.61, 0.98}
\definecolor{arduinoDarkBlue} {rgb} {0.0 , 0.2 , 0.5 }
\definecolor{light-gray}{gray}{0.95}

%%% Define Arduino Language %%%
\lstdefinelanguage{Arduino}{
  language=C++, % begin with default C++ settings 
%
%
  %%% Keyword Color Group 1 %%%  (called KEYWORD3 by arduino)
  keywordstyle=\color{arduinoGreen},   
  deletekeywords={  % remove all arduino keywords that might be in c++
                break, case, override, final, continue, default, do, else, for, 
                if, return, goto, switch, throw, try, while, setup, loop, export, 
                not, or, and, xor, include, define, elif, else, error, if, ifdef, 
                ifndef, pragma, warning,
                HIGH, LOW, INPUT, INPUT_PULLUP, OUTPUT, DEC, BIN, HEX, OCT, PI, 
                HALF_PI, TWO_PI, LSBFIRST, MSBFIRST, CHANGE, FALLING, RISING, 
                DEFAULT, EXTERNAL, INTERNAL, INTERNAL1V1, INTERNAL2V56, LED_BUILTIN, 
                LED_BUILTIN_RX, LED_BUILTIN_TX, DIGITAL_MESSAGE, FIRMATA_STRING, 
                ANALOG_MESSAGE, REPORT_DIGITAL, REPORT_ANALOG, SET_PIN_MODE, 
                SYSTEM_RESET, SYSEX_START, auto, int8_t, int16_t, int32_t, int64_t, 
                uint8_t, uint16_t, uint32_t, uint64_t, char16_t, char32_t, operator, 
                enum, delete, bool, boolean, byte, char, const, false, float, double, 
                null, NULL, int, long, new, private, protected, public, short, 
                signed, static, volatile, String, void, true, unsigned, word, array, 
                sizeof, dynamic_cast, typedef, const_cast, struct, static_cast, union, 
                friend, extern, class, reinterpret_cast, register, explicit, inline, 
                _Bool, complex, _Complex, _Imaginary, atomic_bool, atomic_char, 
                atomic_schar, atomic_uchar, atomic_short, atomic_ushort, atomic_int, 
                atomic_uint, atomic_long, atomic_ulong, atomic_llong, atomic_ullong, 
                virtual, PROGMEM,
                Serial, Serial1, Serial2, Serial3, SerialUSB, Keyboard, Mouse,
                abs, acos, asin, atan, atan2, ceil, constrain, cos, degrees, exp, 
                floor, log, map, max, min, radians, random, randomSeed, round, sin, 
                sq, sqrt, tan, pow, bitRead, bitWrite, bitSet, bitClear, bit, 
                highByte, lowByte, analogReference, analogRead, 
                analogReadResolution, analogWrite, analogWriteResolution, 
                attachInterrupt, detachInterrupt, digitalPinToInterrupt, delay, 
                delayMicroseconds, digitalWrite, digitalRead, interrupts, millis, 
                micros, noInterrupts, noTone, pinMode, pulseIn, pulseInLong, shiftIn, 
                shiftOut, tone, yield, Stream, begin, end, peek, read, print, 
                println, available, availableForWrite, flush, setTimeout, find, 
                findUntil, parseInt, parseFloat, readBytes, readBytesUntil, readString, 
                readStringUntil, trim, toUpperCase, toLowerCase, charAt, compareTo, 
                concat, endsWith, startsWith, equals, equalsIgnoreCase, getBytes, 
                indexOf, lastIndexOf, length, replace, setCharAt, substring, 
                toCharArray, toInt, press, release, releaseAll, accept, click, move, 
                isPressed, isAlphaNumeric, isAlpha, isAscii, isWhitespace, isControl, 
                isDigit, isGraph, isLowerCase, isPrintable, isPunct, isSpace, 
                isUpperCase, isHexadecimalDigit, 
                }, 
  morekeywords={   % add arduino structures to group 1
                break, case, override, final, continue, default, do, else, for, 
                if, return, goto, switch, throw, try, while, setup, loop, export, 
                not, or, and, xor, include, define, elif, else, error, if, ifdef, 
                ifndef, pragma, warning,
                }, 
% 
%
  %%% Keyword Color Group 2 %%%  (called LITERAL1 by arduino)
  keywordstyle=[2]\color{arduinoBlue},   
  keywords=[2]{   % add variables and dataTypes as 2nd group  
                HIGH, LOW, INPUT, INPUT_PULLUP, OUTPUT, DEC, BIN, HEX, OCT, PI, 
                HALF_PI, TWO_PI, LSBFIRST, MSBFIRST, CHANGE, FALLING, RISING, 
                DEFAULT, EXTERNAL, INTERNAL, INTERNAL1V1, INTERNAL2V56, LED_BUILTIN, 
                LED_BUILTIN_RX, LED_BUILTIN_TX, DIGITAL_MESSAGE, FIRMATA_STRING, 
                ANALOG_MESSAGE, REPORT_DIGITAL, REPORT_ANALOG, SET_PIN_MODE, 
                SYSTEM_RESET, SYSEX_START, auto, int8_t, int16_t, int32_t, int64_t, 
                uint8_t, uint16_t, uint32_t, uint64_t, char16_t, char32_t, operator, 
                enum, delete, bool, boolean, byte, char, const, false, float, double, 
                null, NULL, int, long, new, private, protected, public, short, 
                signed, static, volatile, String, void, true, unsigned, word, array, 
                sizeof, dynamic_cast, typedef, const_cast, struct, static_cast, union, 
                friend, extern, class, reinterpret_cast, register, explicit, inline, 
                _Bool, complex, _Complex, _Imaginary, atomic_bool, atomic_char, 
                atomic_schar, atomic_uchar, atomic_short, atomic_ushort, atomic_int, 
                atomic_uint, atomic_long, atomic_ulong, atomic_llong, atomic_ullong, 
                virtual, PROGMEM,
                },  
% 
%
  %%% Keyword Color Group 3 %%%  (called KEYWORD1 by arduino)
  keywordstyle=[3]\bfseries\color{arduinoOrange},
  keywords=[3]{  % add built-in functions as a 3rd group
                Serial, Serial1, Serial2, Serial3, SerialUSB, Keyboard, Mouse,
                },      
%
%
  %%% Keyword Color Group 4 %%%  (called KEYWORD2 by arduino)
  keywordstyle=[4]\color{arduinoOrange},
  keywords=[4]{  % add more built-in functions as a 4th group
                abs, acos, asin, atan, atan2, ceil, constrain, cos, degrees, exp, 
                floor, log, map, max, min, radians, random, randomSeed, round, sin, 
                sq, sqrt, tan, pow, bitRead, bitWrite, bitSet, bitClear, bit, 
                highByte, lowByte, analogReference, analogRead, 
                analogReadResolution, analogWrite, analogWriteResolution, 
                attachInterrupt, detachInterrupt, digitalPinToInterrupt, delay, 
                delayMicroseconds, digitalWrite, digitalRead, interrupts, millis, 
                micros, noInterrupts, noTone, pinMode, pulseIn, pulseInLong, shiftIn, 
                shiftOut, tone, yield, Stream, begin, end, peek, read, print, 
                println, available, availableForWrite, flush, setTimeout, find, 
                findUntil, parseInt, parseFloat, readBytes, readBytesUntil, readString, 
                readStringUntil, trim, toUpperCase, toLowerCase, charAt, compareTo, 
                concat, endsWith, startsWith, equals, equalsIgnoreCase, getBytes, 
                indexOf, lastIndexOf, length, replace, setCharAt, substring, 
                toCharArray, toInt, press, release, releaseAll, accept, click, move, 
                isPressed, isAlphaNumeric, isAlpha, isAscii, isWhitespace, isControl, 
                isDigit, isGraph, isLowerCase, isPrintable, isPunct, isSpace, 
                isUpperCase, isHexadecimalDigit, 
                },      
%
  extendedchars=true,
%
  %%% Set Other Colors %%%
  stringstyle=\color{arduinoDarkBlue},
  showstringspaces=false,  
  commentstyle=\color{arduinoGrey},    
%          
%   
  %%%% Line Numbering %%%%
  numbers=left,                    
  numbersep=5pt,                   
  numberstyle=\color{arduinoGrey},    
  %stepnumber=2,                      % show every 2 line numbers
%
%
  %%%% Code Box Style %%%%
  breaklines=true,                    % wordwrapping
  tabsize=2,         
  basicstyle=\fontsize{9}{11}\ttfamily,
  backgroundcolor=\color{light-gray},
  xleftmargin=.35in
}                                             %%%
%%% somewhere before \begin{document} in your latex file.                      %%%
%%%                                                                            %%%
%%% In your document, place your arduino code between:                         %%%
%%%   \begin{lstlisting}[language=Arduino]                                     %%%
%%% and:                                                                       %%%
%%%   \end{lstlisting}                                                         %%%
%%%                                                                            %%%
%%% Or create your own style to add non-built-in functions and variables.      %%%
%%%                                                                            %%%
 %%%%%%%%%%%%%%%%%%%%%%%%%%%%%%%%%%%%%%%%%%%%%%%%%%%%%%%%%%%%%%%%%%%%%%%%%%%%%%%% 

\usepackage{color}
\usepackage{listings}    
\usepackage{courier}
\usepackage{listingsutf8}

%%% Define Custom IDE Colors %%%
\definecolor{arduinoGreen}    {rgb} {0.17, 0.43, 0.01}
\definecolor{arduinoGrey}     {rgb} {0.47, 0.47, 0.33}
\definecolor{arduinoOrange}   {rgb} {0.8 , 0.4 , 0   }
\definecolor{arduinoBlue}     {rgb} {0.01, 0.61, 0.98}
\definecolor{arduinoDarkBlue} {rgb} {0.0 , 0.2 , 0.5 }
\definecolor{light-gray}{gray}{0.95}

%%% Define Arduino Language %%%
\lstdefinelanguage{Arduino}{
  language=C++, % begin with default C++ settings 
%
%
  %%% Keyword Color Group 1 %%%  (called KEYWORD3 by arduino)
  keywordstyle=\color{arduinoGreen},   
  deletekeywords={  % remove all arduino keywords that might be in c++
                break, case, override, final, continue, default, do, else, for, 
                if, return, goto, switch, throw, try, while, setup, loop, export, 
                not, or, and, xor, include, define, elif, else, error, if, ifdef, 
                ifndef, pragma, warning,
                HIGH, LOW, INPUT, INPUT_PULLUP, OUTPUT, DEC, BIN, HEX, OCT, PI, 
                HALF_PI, TWO_PI, LSBFIRST, MSBFIRST, CHANGE, FALLING, RISING, 
                DEFAULT, EXTERNAL, INTERNAL, INTERNAL1V1, INTERNAL2V56, LED_BUILTIN, 
                LED_BUILTIN_RX, LED_BUILTIN_TX, DIGITAL_MESSAGE, FIRMATA_STRING, 
                ANALOG_MESSAGE, REPORT_DIGITAL, REPORT_ANALOG, SET_PIN_MODE, 
                SYSTEM_RESET, SYSEX_START, auto, int8_t, int16_t, int32_t, int64_t, 
                uint8_t, uint16_t, uint32_t, uint64_t, char16_t, char32_t, operator, 
                enum, delete, bool, boolean, byte, char, const, false, float, double, 
                null, NULL, int, long, new, private, protected, public, short, 
                signed, static, volatile, String, void, true, unsigned, word, array, 
                sizeof, dynamic_cast, typedef, const_cast, struct, static_cast, union, 
                friend, extern, class, reinterpret_cast, register, explicit, inline, 
                _Bool, complex, _Complex, _Imaginary, atomic_bool, atomic_char, 
                atomic_schar, atomic_uchar, atomic_short, atomic_ushort, atomic_int, 
                atomic_uint, atomic_long, atomic_ulong, atomic_llong, atomic_ullong, 
                virtual, PROGMEM,
                Serial, Serial1, Serial2, Serial3, SerialUSB, Keyboard, Mouse,
                abs, acos, asin, atan, atan2, ceil, constrain, cos, degrees, exp, 
                floor, log, map, max, min, radians, random, randomSeed, round, sin, 
                sq, sqrt, tan, pow, bitRead, bitWrite, bitSet, bitClear, bit, 
                highByte, lowByte, analogReference, analogRead, 
                analogReadResolution, analogWrite, analogWriteResolution, 
                attachInterrupt, detachInterrupt, digitalPinToInterrupt, delay, 
                delayMicroseconds, digitalWrite, digitalRead, interrupts, millis, 
                micros, noInterrupts, noTone, pinMode, pulseIn, pulseInLong, shiftIn, 
                shiftOut, tone, yield, Stream, begin, end, peek, read, print, 
                println, available, availableForWrite, flush, setTimeout, find, 
                findUntil, parseInt, parseFloat, readBytes, readBytesUntil, readString, 
                readStringUntil, trim, toUpperCase, toLowerCase, charAt, compareTo, 
                concat, endsWith, startsWith, equals, equalsIgnoreCase, getBytes, 
                indexOf, lastIndexOf, length, replace, setCharAt, substring, 
                toCharArray, toInt, press, release, releaseAll, accept, click, move, 
                isPressed, isAlphaNumeric, isAlpha, isAscii, isWhitespace, isControl, 
                isDigit, isGraph, isLowerCase, isPrintable, isPunct, isSpace, 
                isUpperCase, isHexadecimalDigit, 
                }, 
  morekeywords={   % add arduino structures to group 1
                break, case, override, final, continue, default, do, else, for, 
                if, return, goto, switch, throw, try, while, setup, loop, export, 
                not, or, and, xor, include, define, elif, else, error, if, ifdef, 
                ifndef, pragma, warning,
                }, 
% 
%
  %%% Keyword Color Group 2 %%%  (called LITERAL1 by arduino)
  keywordstyle=[2]\color{arduinoBlue},   
  keywords=[2]{   % add variables and dataTypes as 2nd group  
                HIGH, LOW, INPUT, INPUT_PULLUP, OUTPUT, DEC, BIN, HEX, OCT, PI, 
                HALF_PI, TWO_PI, LSBFIRST, MSBFIRST, CHANGE, FALLING, RISING, 
                DEFAULT, EXTERNAL, INTERNAL, INTERNAL1V1, INTERNAL2V56, LED_BUILTIN, 
                LED_BUILTIN_RX, LED_BUILTIN_TX, DIGITAL_MESSAGE, FIRMATA_STRING, 
                ANALOG_MESSAGE, REPORT_DIGITAL, REPORT_ANALOG, SET_PIN_MODE, 
                SYSTEM_RESET, SYSEX_START, auto, int8_t, int16_t, int32_t, int64_t, 
                uint8_t, uint16_t, uint32_t, uint64_t, char16_t, char32_t, operator, 
                enum, delete, bool, boolean, byte, char, const, false, float, double, 
                null, NULL, int, long, new, private, protected, public, short, 
                signed, static, volatile, String, void, true, unsigned, word, array, 
                sizeof, dynamic_cast, typedef, const_cast, struct, static_cast, union, 
                friend, extern, class, reinterpret_cast, register, explicit, inline, 
                _Bool, complex, _Complex, _Imaginary, atomic_bool, atomic_char, 
                atomic_schar, atomic_uchar, atomic_short, atomic_ushort, atomic_int, 
                atomic_uint, atomic_long, atomic_ulong, atomic_llong, atomic_ullong, 
                virtual, PROGMEM,
                },  
% 
%
  %%% Keyword Color Group 3 %%%  (called KEYWORD1 by arduino)
  keywordstyle=[3]\bfseries\color{arduinoOrange},
  keywords=[3]{  % add built-in functions as a 3rd group
                Serial, Serial1, Serial2, Serial3, SerialUSB, Keyboard, Mouse,
                },      
%
%
  %%% Keyword Color Group 4 %%%  (called KEYWORD2 by arduino)
  keywordstyle=[4]\color{arduinoOrange},
  keywords=[4]{  % add more built-in functions as a 4th group
                abs, acos, asin, atan, atan2, ceil, constrain, cos, degrees, exp, 
                floor, log, map, max, min, radians, random, randomSeed, round, sin, 
                sq, sqrt, tan, pow, bitRead, bitWrite, bitSet, bitClear, bit, 
                highByte, lowByte, analogReference, analogRead, 
                analogReadResolution, analogWrite, analogWriteResolution, 
                attachInterrupt, detachInterrupt, digitalPinToInterrupt, delay, 
                delayMicroseconds, digitalWrite, digitalRead, interrupts, millis, 
                micros, noInterrupts, noTone, pinMode, pulseIn, pulseInLong, shiftIn, 
                shiftOut, tone, yield, Stream, begin, end, peek, read, print, 
                println, available, availableForWrite, flush, setTimeout, find, 
                findUntil, parseInt, parseFloat, readBytes, readBytesUntil, readString, 
                readStringUntil, trim, toUpperCase, toLowerCase, charAt, compareTo, 
                concat, endsWith, startsWith, equals, equalsIgnoreCase, getBytes, 
                indexOf, lastIndexOf, length, replace, setCharAt, substring, 
                toCharArray, toInt, press, release, releaseAll, accept, click, move, 
                isPressed, isAlphaNumeric, isAlpha, isAscii, isWhitespace, isControl, 
                isDigit, isGraph, isLowerCase, isPrintable, isPunct, isSpace, 
                isUpperCase, isHexadecimalDigit, 
                },      
%
  extendedchars=true,
%
  %%% Set Other Colors %%%
  stringstyle=\color{arduinoDarkBlue},
  showstringspaces=false,  
  commentstyle=\color{arduinoGrey},    
%          
%   
  %%%% Line Numbering %%%%
  numbers=left,                    
  numbersep=5pt,                   
  numberstyle=\color{arduinoGrey},    
  %stepnumber=2,                      % show every 2 line numbers
%
%
  %%%% Code Box Style %%%%
  breaklines=true,                    % wordwrapping
  tabsize=2,         
  basicstyle=\fontsize{9}{11}\ttfamily,
  backgroundcolor=\color{light-gray},
  xleftmargin=.35in
}    % adds the arduino language listing
\definecolor{commentgreen}{RGB}{2,112,10}
\definecolor{eminence}{RGB}{108,48,130}
\definecolor{weborange}{RGB}{255,165,0}
\definecolor{frenchplum}{RGB}{129,20,83}


%% Define an Arduino style fore use later %%
\lstdefinestyle{myArduino}{
  language=Arduino,
    %% Add other words needing highlighting below %%
    morekeywords=[1]{},                  % [1] -> dark green
    morekeywords=[2]{FILE_WRITE},        % [2] -> light blue
    morekeywords=[3]{SD, File},          % [3] -> bold orange
    morekeywords=[4]{open, exists, write, SoftwareSerial},      % [4] -> orange
    frame=tb,    
    inputencoding=utf8,
    extendedchars=true,
    literate={è}{{\`{e}}}{1},
    breaklines=true,  
}

\lstdefinestyle{mycpp}{
    language=C++,
    inputencoding=utf8,
    extendedchars=true,
    literate={è}{{\`{e}}}{1},
    frame=tb,
    tabsize=2,
    breaklines=true,                    % wordwrapping
    postbreak=\mbox{\textcolor{red}{$\hookrightarrow$}\space},
    tabsize=2,         
    basicstyle=\fontsize{9}{11}\ttfamily,
    backgroundcolor=\color{light-gray},
    xleftmargin=.25in,
    showstringspaces=false,
    numbers=left,                    
    numbersep=5pt,                   
    %numberstyle=\color{arduinoGrey},    
    %stepnumber=2, 
    %upquote=true,
    commentstyle=\color{commentgreen},
    keywordstyle=\color{eminence},
    stringstyle=\color{red},
    basicstyle=\small\ttfamily, % basic font setting
    emph={int,char,double,float,unsigned,void,bool},
    emphstyle={\color{blue}},
    escapechar=\&,
    % keyword highlighting
    classoffset=1, % starting new class
    otherkeywords={>,<,.,;,-,!,=,~},
    morekeywords={>,<,.,;,-,!,=,~},
    keywordstyle=\color{weborange},
    classoffset=0,
}
%%%%%%%%%%%%%%%%%%%%%%%
\usepackage{siunitx} %pacchetto per le unita' di misura

%%%%%%%%%%%%%%%%%%%%%% per i flowchart
\usepackage{xcolor}
\usepackage{tikz}
\usetikzlibrary{shapes,arrows}
\usetikzlibrary{arrows.meta}
\tikzset{%
    >={Latex[width=2mm,length=2mm]},
      % Specifications for style of nodes:
         declare/.style = {trapezium,draw=black, minimum width=4cm, minimum height=1cm, 
                                trapezium right angle=-70, trapezium left angle=70,
                                minimum width=4cm, minimum height=1cm,
                                text centered, font=\sffamily},
           start/.style = {ellipse, draw, draw=black, minimum width=4cm, 
                                minimum height=1cm, text centered, font=\sffamily},
            cond/.style = {diamond, aspect=2, draw, draw=black,
                                minimum width=4cm, minimum height=1cm,
                                text centered, font=\sffamily},
            rect/.style = {rectangle, draw, draw=black,
                                minimum width=4cm, minimum height=1cm,
                                text centered, font=\sffamily},
}
%%%%%%%%%%%%%%%%%%%%%%%%%%%%%%%%%%%%%%%
% 

\pagenumbering{arabic}
\pagestyle{plain}

% per non farlo anadre a capo ovunque
\usepackage[none]{hyphenat}
% per togliere gli ident all'inizio dei paragrafi
\setlength{\parindent}{0pt}






\begin{document}

\subsection{programma per l'acquisizione dei dati dal kinect e da arduino}
Questo programma \`e stato usato per acquisire i dati dai dispositivi e scriverli in dei file di testo. Per poter lavorare meglio abbiamo separato il programma principale in diverse librerie con scopi specifici, la lista completa delle librerie usate \`e:
\begin{lstlisting}[style=mycpp, caption=librerie usate, captionpos=b]
// libreria usata da visual studio, da togliere in caso si usi un altro ide
#include "pch.h"

// librerie standard per i file stringhe e altro
#include <cstddef>
#include <cstdlib>
#include <fstream>
#include <iomanip>
#include <iostream>
#include <ostream>
#include <string>
#include <math.h>

// librerie create da noi
#include "real_time.h"
#include "kin_file_manager.h"
#include "ard_file_manager.h"
#include "data_structure.h"

// libreria per usare i thread
#include <boost/thread.hpp>

// altre librerie per log, alcuni tipi di dato, un buffer FIFO
#include <boost/log/core.hpp>
#include <boost/call_traits.hpp>
#include <boost/circular_buffer.hpp>
#include <boost/container/vector.hpp>

// altre librerie per il logger
#include <boost/log/attributes.hpp>
#include <boost/log/attributes/scoped_attribute.hpp>
#include <boost/log/expressions.hpp>
#include <boost/log/sinks/sync_frontend.hpp>
#include <boost/log/sinks/text_ostream_backend.hpp>
#include <boost/log/sources/basic_logger.hpp>
#include <boost/log/sources/record_ostream.hpp>
#include <boost/log/sources/severity_logger.hpp>
#include <boost/log/utility/setup/common_attributes.hpp>
#include <boost/log/utility/setup/console.hpp>
#include <boost/log/utility/setup/file.hpp>

#include <boost/smart_ptr/make_shared_object.hpp>
#include <boost/smart_ptr/shared_ptr.hpp>

// altre librerie per i thread 
#include <boost/thread/condition_variable.hpp>
#include <boost/thread/mutex.hpp>
#include <boost/thread/thread.hpp>

// librereria per il tempo
#include <boost/date_time/posix_time/posix_time.hpp>

// libreria fatta da noi per gestire la seriale
#include "Serial_handler.h"

// librerie di Windows e kinect 
#include <Windows.h>
#include <Kinect.h>
#include <Shlobj.h>

// namespace standard
using namespace std;
\end{lstlisting}
%
%
Le librerie fatte da noi sono: "real\_time.h", "kin\_file\_manager.h", "ard\_file\_manager.h", "data\_structure.h", "Serial\_handler.h". 
In "real\_time.h" ci sono alcune funzioni per prendere il tempo e misurarlo:
\begin{lstlisting}[style=mycpp, caption=librerie usate, captionpos=b]
#pragma once

#ifndef _REAL_TIME_H_
#define _REAL_TIME_H_

#include <boost/chrono.hpp>
#include <boost/timer/timer.hpp>

class real_time
{
private:
  boost::timer::cpu_timer timer;
  boost::timer::cpu_times t;

public:
  /// <summary>
  /// start the time counting
  /// </summary>
  /// <returns></returns>
  void start();

  /// <summary>
  /// return enlapsed time in millisecond
  /// </summary>
  /// <returns></returns>
  float stop();

  /// <summary>
  /// get the time in millisecond
  /// </summary>
  /// <returns></returns>
  uint64_t get_curr_time();

};

#endif // #ifndef _REAL_TIME_H_
\end{lstlisting}
%
%
In "kin\_file\_manager.h" ci sono le funzioni per gestire il file di testo relativo al kinect:
\begin{lstlisting}[style=mycpp, caption=librerie usate, captionpos=b]
#pragma once

#ifndef _KIN_FILE_MANAGER_H_
#define _KIN_FILE_MANAGER_H_

#include <iostream>
#include <ostream>
#include <fstream>
#include <string>

// in questa libreria sono dichiarate le varie strutture dati
#include "data_structure.h"

class kin_file_manager
{
private:
  std::fstream f;
  std::ios::_Openmode mode;
  unsigned long int line_id = 0;

public:
  // il costruttore provvede a settare il nome del file e la modalit\`a di apertura
  kin_file_manager(std::string file_name, std::ios::_Openmode mode);

  // questa funzione provvede a scrivere un singolo blocco di dati nel file
  void write_data_line(kinect_data dat);

  // funzione che legge un blocco di dati dal file e lo sposta nella struttura dati
  // se si \`e raggiunta la fine del file la funzione ritorna 0, altrimenti ritorna 1. 
  bool read_data_line(kinect_data* dat);

  // distruttore della classe, chiude il file
  ~kin_file_manager()
  {
  f.close();
  }

  // funzione che chiude il file
  void close()
  {
  f.close();
  }

};

#endif // #ifndef _KIN_FILE_MANAGER_H_
\end{lstlisting}
%
%
Stessa cosa per "ard\_file\_manager.h" che contiene le funzioni per la gestione del file di testo relativo ai dati di arduino:
\begin{lstlisting}[style=mycpp, caption=librerie usate, captionpos=b]
#pragma once

#ifndef _ARD_FILE_MANAGER_
#define _ARD_FILE_MANAGER_

#include <iostream>
#include <ostream>
#include <fstream>
#include <string>
#include "data_structure.h"


class ard_file_manager
{
private:
  std::fstream f;
  std::ios::_Openmode mode;
  unsigned long int line_id = 0;

public:
  ard_file_manager(std::string file_name, std::ios::_Openmode mode);

  void write_data_line(arduino_data dat);

  bool read_data_line(arduino_data* dat);

  ~ard_file_manager()
  {
    f.close();
  }

  void close()
  {
    f.close();
  }
};

#endif //#ifndef _ARD_FILE_MANAGER_
\end{lstlisting}
Questa libreria \`e strutturata in modo totalmente simile alla precedente con l'unica differenza che la struttura dati utilizzata \`e relativa ad arduino e non al kinect.
\\
%
%
La libreria "data\_structure.h" \`e fondamentale e definisce le strutture dati utilizzate nel resto del programma:
\begin{lstlisting}[style=mycpp, caption=librerie usate, captionpos=b]
#pragma once


#ifndef _DATA_STRUCTURE_H_
#define _DATA_STRUCTURE_H_

#include <iostream>
#include <string>
#include <boost/array.hpp>

////////// massiiiiiiiiiiiiiiiiii <------------------------------------------------
// struttura dati per gestire i dati provenienti dal kinect
struct kinect_data
{
  static const int number_of_joints = 3;

  // sottostruttura che contiene i dati di ogni giunto: il suo numero, il giunto attaccato ad esso, la posizione e l'angolo rispetto all'altro giunto
  struct joit_data
  {
    int joint_name = -1;
    int to_joint = -1;
    float position[3] = { 0, 0, 0 };
    float angle[3] = { 0, 0, 0 };
  };

  // <-------------------------------------------------------massiiii
  int needed_joint[number_of_joints] = { 8, 9, 10 };
  int to_joint[24] = { -1, -1, -1, -1, -1, -1, -1, -1, 9, 10, -1, -1, -1, -1, -1, -1, -1, -1, -1, -1, -1, -1, -1, -1 };
  int to_ref_joint[24] = { -1, -1, -1, -1, -1, -1, -1, -1, 0, 1, 2, -1, -1, -1, -1, -1, -1, -1, -1, -1, -1, -1, -1, -1 }; // tabella di accesso (chiave numero joint kinect) -> (restituisce l-indice del numero del giunto in needed_joint)    
  float ref_Ang_x[number_of_joints] = { 0, 0, 0 }; //per  8, 9
  float ref_Ang_y[number_of_joints] = { 0, 0, 0 }; //per 8, 9
  float ref_Ang_z[number_of_joints] = { 180, 180, 0 }; //per 8, 9
  float mul_fctor_x[number_of_joints] = { 0, 0, 0 }; //per 8, 9
  float mul_fctor_y[number_of_joints] = { 1, 1, 0 }; //per 8, 9
  float mul_fctor_z[number_of_joints] = { 1, 1, 0 }; //per 8, 9

  //int needed_joint[number_of_joints] = { 4, 5, 6, 8, 9, 10, 1, 20 };
  //int to_joint[24] = { -1, 20, -1, -1, 5, 6, -1, -1, 9, 10, -1, -1, -1, -1, -1, -1, -1, -1, -1, -1, -1, -1, -1, -1, };
  //int to_ref_joint[24] = { -1, 6, -1, -1, -1, 1, 2, -1, 3, 4, 5, -1, -1, -1, -1, -1, -1, -1, -1, -1, 7, -1, -1, -1 }; // tabella di accesso (chiave numero joint kinect)  
  //float ref_Ang_x[number_of_joints] = { 0, 0, 0, 0, 0, 0, -90, 0 }; //per 4, 5, 8, 9
  //float ref_Ang_y[number_of_joints] = { 0, 0, 0, 0, 0, 0, 90, 0 }; //per 4, 5, 8, 9
  //float ref_Ang_z[number_of_joints] = { 180, 180, 0, 180, 180, 0, 270, 0 }; //per 4, 5, 8, 9
  //float mul_fctor_x[number_of_joints] = { 0, 0, 0, 0, 0, 0, 0, 0 }; //per 4, 5, 8, 9
  //float mul_fctor_y[number_of_joints] = { 1, 1, 0, 1, 1, 0, 0, 0 }; //per 4, 5, 8, 9
  //float mul_fctor_z[number_of_joints] = { 1, 1, 0, 1, 1, 0, 0, 0 }; //per 4, 5, 8, 9

  joit_data needed_joints[number_of_joints];

  // tempo esatto in cui i dati sono stati acquisiti
  uint64_t frame_time = 0;

  // questo che era? <-----------------------------------
  uint64_t contatore = 0;

  // funzione per verificare se il frame acquisito \`e parziale o completo 
  bool full_frame();

  // funzione per stampare i dati memorizzati
  void print_data();

  float jointAngleX(float *P1, float *P2);

  float jointAngleY(float *P1, float *P2);

  float jointAngleZ(float *P1, float *P2);

  void updateAngles();

  // funzione per convertire il numero del giunto in una stringa
  std::string joint_Enum_ToStr(int n, std::string language);
};

// struttura dati per gestire i dati provenienti da arduino
struct arduino_data
{
  // variabili che rappresentano i dati: dell'accelerometro, del giroscopio, del magnetometro e la temperatura
  float acc_xyz[3];
  float gy_xyz[3];
  float magn_xyz[3];
  float temp;

  // tempo esatto di acquisizione dei dati
  uint64_t frame_time = 0;

  // ?
  uint64_t contatore = 0;

  // funzione che permette di stampare il set di dati attualmente immagazzinato a schermo 
  void print_data();
};

// struttura template che gestisce un set di dati del dataset
template<typename type_in, std::size_t N, typename type_out, std::size_t M>
struct dataset_data
{
  boost::array<type_in, N> in;
  boost::array<type_out, M> out;

  // stamapa i dati attualmente immagazzinati
  void print_data();
};

#endif //#ifndef _DATA_STRUCTURE_H_
\end{lstlisting}
%
%
Infine l'ultima libreria fatta da noi ospita la classe che gestisce la seriale per la comunicazione tramite bluetooth con arduino:
\begin{lstlisting}[style=mycpp, caption=librerie usate, captionpos=b]
#pragma once

#ifndef _SERIAL_HADLER_H_
#define _SERIAL_HADLER_H_

#include <iostream>
#include <string>
#include <boost/asio.hpp> 
#include <boost/bind.hpp>
#include <boost/asio/serial_port.hpp> 


class Serial
{
private:
  boost::asio::serial_port* port;

  union Scomp_float
  {
    float n_float;
    uint32_t n_int;
    uint8_t n_bytes[4];
  };

  union Scomp_msg_tag
  {
    uint16_t n_16b;
    uint8_t n_bytes[2];
  };

public:


  Serial(std::string com);


  /// <summary>
  /// sincronizza arduino ed il pc
  /// </summary>
  void sinc();

  /// <summary>
  /// invia un floar ad arduino
  /// </summary>
  /// <param name="n"></param>
  void send_float(float n);

  /// <summary>
  /// riceve un float da arduino
  /// </summary>
  /// <returns></returns>
  float receive_float();

  void send_char(char ch);

  char receive_char();

  /// <summary>
  /// receive the data of acc, gy, magn from arduino
  /// </summary>
  /// <param name="acc_xyz"></param>
  /// <param name="g_xyz"></param>
  /// <param name="magn"></param>
  /// <param name="temp"></param>
  /// <returns>
  /// -> 0 if the trasmission is good
  /// -> 1 if it falied
  /// </returns>
  int receive_data(float* acc_xyz, float* g_xyz, float* magn, float* temp);

  ~Serial()
  {
    port->close();
  }
};

#endif // #ifndef _SERIAL_HADLER_H_
\end{lstlisting}
%
%
ora devo cominciare con il main e tutte le classi che ci stanno dentro

\end{document}